\documentclass[crop,tikz]{standalone}
\usepackage{mathtools}
\usepackage{soul} % Similar package to ulem

% Core Mathematics
\usepackage{siunitx}

% circuitikz
\usepackage{xstring} % CIRCUITIKZ DEPENDENCY
\usepackage[american]{circuitikz}

\usetikzlibrary{
    arrows,
    arrows.meta,
    matrix
}

\newcommand\rhs{\textit{RHS}}
\newcommand\lhs{\textit{LHS}}

\newcommand{\Nth}[2]{\ensuremath{{#1}^\textit{#2}}}
\newcommand{\diff}{\ensuremath{\mathrm{d}}}
\newcommand{\parallelsum}{\mathbin{\|}}

\newcommand{\myul}[1]{\ul{#1}} % A proxy for the preferred underline command.

\DeclarePairedDelimiter\parens{(}{)}
\DeclarePairedDelimiter\brackets{[}{]}
\DeclarePairedDelimiter\braces{\{}{\}}

\DeclarePairedDelimiter\abs{\lvert{}}{\rvert{}}
\DeclarePairedDelimiter\inte{[}{]}

\definecolor{myblue}{RGB}{66,133,244}
\definecolor{myteal}{RGB}{70,189,198}
\definecolor{mygreen}{RGB}{52,168,83}
\definecolor{myyellow}{RGB}{251,188,4}
\definecolor{myorange}{RGB}{255,109,1}
\definecolor{myred}{RGB}{234,67,53}
\definecolor{mypurple}{HTML}{9900FF}

\definecolor{mycontrastblue}{HTML}{0000FF}
\definecolor{mycontrastred}{HTML}{FF0000}
\definecolor{mycontrastpurple}{HTML}{9900FF}

% TikZ styles
\colorlet{DiagramGrayFill}{black!10}      \colorlet{DiagramGrayBorder}{black!60}
\colorlet{DiagramYellowFill}{myyellow!40} \colorlet{DiagramYellowBorder}{myyellow}
\colorlet{DiagramBlueFill}{myblue!30}     \colorlet{DiagramBlueBorder}{myblue}
\colorlet{DiagramRedFill}{myred!30}       \colorlet{DiagramRedBorder}{myred}
\colorlet{DiagramGreenFill}{mygreen!30}   \colorlet{DiagramGreenBorder}{mygreen}
\colorlet{DiagramPurpleFill}{mypurple!35} \colorlet{DiagramPurpleBorder}{mypurple}
\colorlet{DiagramOrangeFill}{myorange!40} \colorlet{DiagramOrangeBorder}{myorange}
\tikzstyle{GraphNode} = [
    circle,
    minimum size=8mm,
    text centered,
    draw=DiagramGrayBorder,
    line width=1.6pt,
    fill=DiagramGrayFill
]
\tikzstyle{GraphNodeSmall} = [
    GraphNode, draw=DiagramRedBorder, fill=DiagramRedFill, minimum size=6mm
]
\tikzstyle{GraphNodeYellow} = [
    GraphNode, draw=DiagramYellowBorder, fill=DiagramYellowFill
]
\tikzstyle{GraphNodeBlue} = [
    GraphNode, draw=DiagramBlueBorder, fill=DiagramBlueFill
]
\tikzstyle{GraphNodeRed} = [
    GraphNode, draw=DiagramRedBorder, fill=DiagramRedFill
]
\tikzstyle{GraphNodeGreen} = [
    GraphNode, draw=DiagramGreenBorder, fill=DiagramGreenFill
]
\tikzstyle{GraphNodePurple} = [
    GraphNode, draw=DiagramPurpleBorder, fill=DiagramPurpleFill
]
\tikzstyle{GraphNodeOrange} = [
    GraphNode, draw=DiagramOrangeBorder, fill=DiagramOrangeFill
]

%%%%%%%%%%%%%%%%%%%%%%%%%%%%%%%%%%%%%%%%%%%%%%%%%%%%%%%%%%%%%%%%%%%%%
% Circuitikz %%%%%%%%%%%%%%%%%%%%%%%%%%%%%%%%%%%%%%%%%%%%%%%%%%%%%%%%
%%%%%%%%%%%%%%%%%%%%%%%%%%%%%%%%%%%%%%%%%%%%%%%%%%%%%%%%%%%%%%%%%%%%%

% Custom Circuitikz options
\ctikzset{tripoles/mos style/arrows} % Adds arrows to MOSFETs
\ctikzset{label/align=straight} % Straightens diagonal labels

% Circuitikz part proportions
% Based on: https://github.com/circuitikz/circuitikz/blob/4feb150ba25cd7ad80c94a56f5d5d516200bfbf0/tex/circuitikz.sty#L258

\ctikzset{monopoles/ground/width/.initial=.15}

\ctikzset{bipoles/resistor/height/.initial=.23}
\ctikzset{bipoles/resistor/width/.initial=.6}
\ctikzset{bipoles/capacitor/height/.initial=.4}
\ctikzset{bipoles/capacitor/width/.initial=.1}
\ctikzset{bipoles/potentiometer/height/.initial=.6}
\ctikzset{bipoles/potentiometer/height 2/.initial=.23}
\ctikzset{bipoles/potentiometer/width/.initial=.6}
\ctikzset{bipoles/photoresistor/height/.initial=.6}
\ctikzset{bipoles/photoresistor/height 2/.initial=.23}
\ctikzset{bipoles/photoresistor/width/.initial=.6}
\ctikzset{bipoles/thermistor/main/.initial=.7}
\ctikzset{bipoles/thermistor/height/.initial=.328}%.23/.7
\ctikzset{bipoles/thermistor/width/.initial=.6}
\ctikzset{bipoles/thermistorntc/width/.initial=.6}
\ctikzset{bipoles/thermistorntc/main/.initial=.7}
\ctikzset{bipoles/thermistorntc/height/.initial=.328}%.23/.7
\ctikzset{bipoles/thermistorntc/height 2/.initial=.75}%.23/.7
\ctikzset{bipoles/thermistorptc/width/.initial=.6}
\ctikzset{bipoles/thermistorptc/main/.initial=.7}
\ctikzset{bipoles/thermistorptc/height/.initial=.328}%.23/.7
\ctikzset{bipoles/varistor/main/.initial=.7}
\ctikzset{bipoles/varistor/height/.initial=.328}%.23/.7
\ctikzset{bipoles/varistor/width/.initial=.6}
\ctikzset{bipoles/vresistor/height/.initial=.45}
\ctikzset{bipoles/vresistor/width/.initial=.6}
\ctikzset{bipoles/generic/height/.initial=.23}
\ctikzset{bipoles/generic/width/.initial=.6}
\ctikzset{bipoles/generic potentiometer/height/.initial=.6}
\ctikzset{bipoles/generic potentiometer/height 2/.initial=.23}
\ctikzset{bipoles/generic potentiometer/width/.initial=.6}
\ctikzset{bipoles/ageneric/height/.initial=.23}
\ctikzset{bipoles/ageneric/width/.initial=.6}
\ctikzset{bipoles/memristor/height/.initial=.23}
\ctikzset{bipoles/memristor/wave height/.initial=.375}
\ctikzset{bipoles/memristor/width/.initial=.60}
\ctikzset{bipoles/tgeneric/height/.initial=.525}
\ctikzset{bipoles/tgeneric/width/.initial=.6}
\ctikzset{bipoles/tfullgeneric/height/.initial=.525}
\ctikzset{bipoles/tfullgeneric/width/.initial=.60}
\ctikzset{bipoles/fullgeneric/height/.initial=.23}
\ctikzset{bipoles/fullgeneric/width/.initial=.6}
\ctikzset{bipoles/diode/height/.initial=.3}
\ctikzset{bipoles/diode/width/.initial=.25}
\ctikzset{bipoles/bidirectionaldiode/height/.initial=.66}
\ctikzset{bipoles/bidirectionaldiode/width/.initial=.6}
\ctikzset{bipoles/bidirectionaldiode/diode width left/.initial=.3}
\ctikzset{bipoles/bidirectionaldiode/diode width right/.initial=.3}

\ctikzset{tripoles/thyristor/height/.initial=.66}
\ctikzset{tripoles/thyristor/height 2/.initial=.3}
\ctikzset{tripoles/thyristor/width/.initial=.6}
\ctikzset{tripoles/thyristor/diode height/.initial=.3}
\ctikzset{tripoles/thyristor/diode width left/.initial=.4}
\ctikzset{tripoles/thyristor/diode width right/.initial=.3}

\ctikzset{tripoles/triac/height/.initial=.66}
\ctikzset{tripoles/triac/width/.initial=.6}
\ctikzset{tripoles/triac/diode width left/.initial=.3}
\ctikzset{tripoles/triac/diode width right/.initial=.3}

% Custom Proportions

\ctikzset{bipoles/crossing/size=.5}

% Reusable Style Macros

% Used for magnetic coupling arrows in electrical circuit diagrams.
\tikzset{simshadows/style/ee/magneticcouplingarrow/.style={latex-latex, magneticcouplingcolor, line width=1.2pt}}

% Used for iron/ferrite cores in electrical circuit diagrams.
\tikzset{simshadows/style/ee/ironcore/.style={line width=2.0pt}}

% Custom TikZ Graphics

% Mesh current indicators
% Example:
%     \draw
%         (0,0) pic {simshadows/ee/meshcw} node {$i_1$}
%     ;
\tikzset{
    simshadows/ee/meshcw/.pic={
        \draw[thick, stealth-]
            (0,0) ++(-70:0.35) arc (-70:190:0.35)
        ;
    },
    simshadows/ee/meshcc/.pic={
        \draw[thick, -stealth]
            (0,0) ++(-10:0.35) arc (-10:250:0.35)
        ;
    },
}

% Transformer core
% (Note: This will need to be drawn manually if a longer/shorter core is desired. In which case, use this pic as a template.)
% Example:
%     \draw
%         (0,0) pic {simshadows/ee/ironcore}
%     ;
\tikzset{
    simshadows/ee/ironcore/.pic={
        \draw[simshadows/style/ee/ironcore]
            ( 0.08,0.55) -- ++(0,-1.1)
            (-0.08,0.55) -- ++(0,-1.1)
        ;
    },
}

% Generic arrows
% Example:
%     \node[simshadows/GenericBlockArrow] {};
%     \node[simshadows/GenericArrow] {};
\tikzset{
    simshadows/GenericBlockArrow/.style={
        single arrow,
        draw,
        minimum width=4ex,
        minimum height=5ex,
        single arrow head extend=1.0ex,
    },
    simshadows/GenericGrayBlockArrow/.style={
        single arrow,
        draw,
        minimum width=4.5ex,
        minimum height=5.5ex,
        single arrow head extend=1.1ex,
        simshadows/style/softgray,
    }
}

\newcommand{\DrawnDownArrow}[1][]{%
    \begin{tikzpicture}[scale=1, transform shape]
        \node[simshadows/GenericBlockArrow, rotate=-90, #1] {};
    \end{tikzpicture}%
}

\newcommand{\DrawnRightArrow}[1][]{%
    \begin{tikzpicture}[scale=1, transform shape]
        \node[simshadows/GenericBlockArrow, #1] {};
    \end{tikzpicture}%
}


\begin{document}
\newcommand{\Y}{0.44}
\newcommand{\Yhalf}{0.22}
\newcommand{\MyTableCell}[3]{%
    (R#1 -| C#2) node[right, align=left, font=\footnotesize] {\vphantom{$M_I^{I^x}$}\texttt{#3}}%
}
\begin{tikzpicture}[scale=1, transform shape]
    \path
        (0,0)      coordinate (LeftEdge)
        ++(0.08,0) coordinate (C1)
        ++(1.98,0) coordinate (C2)
        ++(1.50,0) coordinate (C3)
        ++(2.10,0) coordinate (RightEdge)
        ++(0.20,0) coordinate (RightOS1)
        ++(0.50,0) coordinate (RightOS2)
    ;
    \begin{scope}[shift={(0,0)}]
        \path
            (0,0)         coordinate (Top)
            ++(0,-\Yhalf) coordinate (RA)
            ++(0,-\Yhalf) coordinate (Bottom)

            (Top)
            ++(0, 0.040) % Slight extra offset
            ++(0, \Yhalf) coordinate (RT) % Title row

            (Top)
            ++(0,-0.1) coordinate (TopOS1)

            (Top)
            ++(0,1.2) coordinate (TopOS2)
            ++(0,0.5) coordinate (TopOS3)
        ;
        \draw
            (Top -| LeftEdge) rectangle (RightEdge |- Bottom)
        ;
        \draw
            \MyTableCell{T}{1}{\textbf{Customers}}

            \MyTableCell{A}{1}{customerid}
            \MyTableCell{A}{2}{name}
            \MyTableCell{A}{3}{address}
        ;
    \end{scope}
    \begin{scope}[shift={(0,-0.52)}]
        \path
            (0,0)         coordinate (Top)
            ++(0,-\Yhalf) coordinate (R1)
            ++(0,-\Y)     coordinate (R2)
            ++(0,-\Y)     coordinate (R3) coordinate (HorizontalMid)
            ++(0,-\Y)     coordinate (R4)
            ++(0,-\Y)     coordinate (R5)
            ++(0,-\Yhalf) coordinate (Bottom)
        ;
        \draw
            (Top -| LeftEdge) rectangle (RightEdge |- Bottom)
        ;
        \draw
            \MyTableCell{1}{1}{1774504}
            \MyTableCell{1}{2}{Amy}
            \MyTableCell{1}{3}{Surry Hills}

            \MyTableCell{2}{1}{4389167}
            \MyTableCell{2}{2}{Chris}
            \MyTableCell{2}{3}{Richmond}

            \MyTableCell{3}{1}{4622780}
            \MyTableCell{3}{2}{Josh}
            \MyTableCell{3}{3}{North Ryde}
            
            \MyTableCell{4}{1}{5691729}
            \MyTableCell{4}{2}{Sam}
            \MyTableCell{4}{3}{Kensington}

            \MyTableCell{5}{1}{9527291}
            \MyTableCell{5}{2}{Vanessa}
            \MyTableCell{5}{3}{Richmond}
        ;

        \draw[mycontrastblue]
            (HorizontalMid -| RightOS2)
            ++(0.10,0) coordinate (RowArrowConvg)
            ++(0.05,0) node[right, align=left, font=\small] {
                \textbf{\textit{tuples,}} \\
                \textbf{\textit{rows,}} \\
                \textbf{\textit{records}}
            }
        ;
        \draw[-stealth, cap=round, line width=2.0pt, mycontrastblue] (R1 -| RightOS2) -- (R1 -| RightOS1);
        \draw[-stealth, cap=round, line width=2.0pt, mycontrastblue] (R2 -| RightOS2) -- (R2 -| RightOS1);
        \draw[-stealth, cap=round, line width=2.0pt, mycontrastblue] (R3 -| RightOS2) -- (R3 -| RightOS1);
        \draw[-stealth, cap=round, line width=2.0pt, mycontrastblue] (R4 -| RightOS2) -- (R4 -| RightOS1);
        \draw[-stealth, cap=round, line width=2.0pt, mycontrastblue] (R5 -| RightOS2) -- (R5 -| RightOS1);
        \draw[          cap=round, line width=2.0pt, mycontrastblue] (R1 -| RightOS2) -- (R5 -| RightOS2);
        %\draw[         cap=round, line width=2.0pt, mycontrastblue] (RowArrowConvg) -- (RowArrowConvg -| RightOS2);

        \draw[mycontrastpurple]
            (TopOS2 -| C3)
            ++(1.8, 0) coordinate (ColArrowConvgRef)
            ++(0.05, 0) node[right, align=left, font=\small] {
                \textbf{\textit{attributes,}} \\
                \textbf{\textit{columns,}} \\
                \textbf{\textit{fields}}
            }
        ;
        \path
            (RA -| C1) ++(1.75, 0.30) coordinate (ColArrowEnd1)
            (RA -| C2) ++(0.85, 0.30) coordinate (ColArrowEnd2)
            (RA -| C3) ++(0.80, 0.30) coordinate (ColArrowEnd3)
            (ColArrowConvgRef) ++(0, -0.1) coordinate (ColArrowConvg)
        ;
        \draw[-stealth, cap=round, line width=2.0pt, mycontrastpurple] (ColArrowConvg) -- (ColArrowEnd1);
        \draw[-stealth, cap=round, line width=2.0pt, mycontrastpurple] (ColArrowConvg) -- (ColArrowEnd2);
        \draw[-stealth, cap=round, line width=2.0pt, mycontrastpurple] (ColArrowConvg) -- (ColArrowEnd3);

        \draw[mycontrastred]
            (TopOS3 -| C2)
            ++(0.30, 0) coordinate (RelArrowConvgRef)
            ++(0.05, 0) node[right, align=left, font=\small] {
                \textbf{\textit{relation,}} \\
                \textbf{\textit{table}}
            }
        ;
        \path
            (RT -| C1) ++(1.00, 0.25) coordinate (RelArrowEnd)
            (RelArrowConvgRef) ++(0, -0.3) coordinate (RelArrowConvg)
        ;
        \draw[-stealth, cap=round, line width=2.0pt, mycontrastred] (RelArrowConvg) -- (RelArrowEnd);

        % An attempt was made to make smooth lines...
        %\draw[-stealth, cap=round, line width=1.5pt]
        %    plot [smooth] coordinates { (RowArrowConvg) (R1 -| RightOS2) (R1 -| RightOS1) };
        %\draw[-stealth, cap=round, line width=1.5pt]
        %    plot [smooth] coordinates { (RowArrowConvg) (R2 -| RightOS2) (R2 -| RightOS1) };
        %\draw[-stealth, cap=round, line width=1.5pt]
        %    plot [smooth] coordinates { (RowArrowConvg) (R3 -| RightOS2) (R3 -| RightOS1) };
        %\draw[-stealth, cap=round, line width=1.5pt]
        %    plot [smooth] coordinates { (RowArrowConvg) (R4 -| RightOS2) (R4 -| RightOS1) };
        %\draw[-stealth, cap=round, line width=1.5pt]
        %    plot [smooth] coordinates { (RowArrowConvg) (R5 -| RightOS2) (R5 -| RightOS1) };
    \end{scope}
\end{tikzpicture}

%%
%% THE OTHER EXAMPLE RELATIONS BELOW ARE FOR A MORE EXTENDED EXAMPLE.
%% I AM NOT INCLUDING THEM FOR SPACE REASONS... FOR NOW.
%%

%\medskip

%{\footnotesize%
%\begin{tabular}{|llr|}
%    \multicolumn{3}{l}{\ttd{Accounts}}
%        \\ \hline
%    \multicolumn{1}{|l}{\ttd{branch}}
%        & \multicolumn{1}{l}{\ttd{accountno}}
%        & \multicolumn{1}{l|}{\ttd{balance}}
%        \\ \hline\hline
%    \ttd{Richmond}
%        & \ttd{A-02772}
%        & \ttd{20.87}
%        \\
%    \ttd{Macquarie} % The actual suburb is called Macquarie Park, but I want a smaller table.
%        & \ttd{J-31553}
%        & \ttd{60899.58}
%        \\
%    \ttd{Richmond}
%        & \ttd{W-40018}
%        & \ttd{84731.08}
%        \\
%    \ttd{Haymarket}
%        & \ttd{A-74884}
%        & \ttd{483.94}
%        \\
%    \ttd{Haymarket}
%        & \ttd{P-85953}
%        & \ttd{7294.62}
%        \\ \hline
%\end{tabular}%

%\medskip
%%\hspace{2.0ex} % TODO: Prefer to instead make these two tables side-by-side if possible.

%\begin{tabular}{|ll|}
%    \multicolumn{2}{l}{\ttd{HeldBy}}
%        \\ \hline
%    \multicolumn{1}{|l}{\ttd{accountno}}
%        & \multicolumn{1}{l|}{\ttd{owner}}
%        \\ \hline\hline
%    \ttd{A-02772}
%        & \ttd{4389167}
%        \\
%    \ttd{A-02772}
%        & \ttd{9527291}
%        \\
%    \ttd{J-31553}
%        & \ttd{4622780}
%        \\
%    \ttd{W-40018}
%        & \ttd{4389167}
%        \\
%    \ttd{W-40018}
%        & \ttd{9527291}
%        \\
%    \ttd{A-74884}
%        & \ttd{1774504}
%        \\
%    \ttd{P-85953}
%        & \ttd{1774504}
%        \\ \hline
%\end{tabular}%
%}
\end{document}

