\documentclass{article}

%\usepackage[margin=2.5cm, bottom=3.0cm]{geometry}
\usepackage[margin=1.5cm, bottom=2.0cm]{geometry}

\usepackage{../_latex_includes/sharedpkg}

\def\TITLE{Random Math Solutions}

\setcounter{tocdepth}{3}
\setcounter{secnumdepth}{3}

\begin{document}

\thispagestyle{plain}
\MakeCustomTitle
\bigskip

%%%%%%%%%%%%%%%%%%%%%%%%%%%%%%%%%%%%%%%%%%%%%%%%%%%%%%%%%%%%%%%%%%%%%%%%%%%%%%%%%%%%%%%%%%%%%%%%%%%%
%%%%%%%%%%%%%%%%%%%%%%%%%%%%%%%%%%%%%%%%%%%%%%%%%%%%%%%%%%%%%%%%%%%%%%%%%%%%%%%%%%%%%%%%%%%%%%%%%%%%

\begin{QuestionFrame}
    Considering:
    \begin{equation*}
        \int{\sech{ax} \,\diff{x}}
            = \frac{1}{a} \tan^{-1}{\parens*{\sinh{ax}}}
            = \frac{2}{a} \tan^{-1}{\parens*{\tanh{\frac{ax}{2}}}}
    \end{equation*}
    
    Prove that:
    \begin{equation*}
        \frac{1}{a} \tan^{-1}{\parens*{\sinh{ax}}}
            = \frac{2}{a} \tan^{-1}{\parens*{\tanh{\frac{ax}{2}}}}
    \end{equation*}
\end{QuestionFrame}

Let us use the following identities for this proof:
\begin{gather*}
    \tan^{-1}{x} = 2 \tan^{-1}{\parens*{\frac{x}{1 + \sqrt{1 + x^2}}}} \\
    \sinh{x} = \frac{e^x - e^{-x}}{2} = \frac{e^{2x} - 1}{2 e^x} \\
    \tanh{x} = \frac{e^x - e^{-x}}{e^x + e^{-x}} = \frac{e^{2x} - 1}{e^{2x} + 1}
\end{gather*}

We start by rewriting LHS:
\begin{equation*}
    \lhs
        = \frac{1}{a} \tan^{-1}{\parens*{\sinh{ax}}}
        = \frac{2}{a} \tan^{-1}{\parens*{\frac{\sinh{ax}}{1 + \sqrt{1 + \sinh^2{ax}}}}} 
\end{equation*}

Now, to finish the proof, we will now show the following to be true:
\begin{equation*}
    \tanh{\frac{ax}{2}} = \frac{\sinh{ax}}{1 + \sqrt{1 + \sinh^2{ax}}}
\end{equation*}

To simplify further working, we will change the variable:
\begin{equation*}
    \tanh{x} = \frac{\sinh{2x}}{1 + \sqrt{1 + \sinh^2{2x}}}
\end{equation*}

Starting with the RHS:
\begin{align*}
    \rhs
    &= \frac{\frac{e^{4x} - 1}{2 e^{2x}}}{1 + \sqrt{1 + \parens*{\frac{e^{4x} - 1}{2 e^{2x}}}^2}}
    = \frac{e^{4x} - 1}{2 e^{2x} \parens*{1 + \sqrt{1 + \frac{e^{8x} - 2 e^{4x} + 1}{4 e^{4x}}}}}
    = \frac{e^{4x} - 1}{2 e^{2x} \parens*{1 + \sqrt{\frac{e^{8x} - 2 e^{4x} + 1 + 4 e^{4x}}{4 e^{4x}}}}}
    = \frac{e^{4x} - 1}{2 e^{2x} \parens*{1 + \sqrt{\frac{e^{8x} + 2 e^{4x} + 1}{4 e^{4x}}}}}
    \\
    &= \frac{e^{4x} - 1}{2 e^{2x} \parens*{1 + \sqrt{\frac{\parens*{e^{4x} + 1}^2}{4 e^{4x}}}}}
    = \frac{e^{4x} - 1}{2 e^{2x} \parens*{1 + \frac{e^{4x} + 1}{2 e^{2x}}}}
    = \frac{e^{4x} - 1}{2 e^{2x} + e^{4x} + 1}
    = \frac{\parens*{e^{2x} - 1} \parens*{e^{2x} + 1}}{\parens*{e^{2x} + 1}^2}
    = \frac{e^{2x} - 1}{e^{2x} + 1}
    = \tanh{x}
    = \lhs
\end{align*}

as required.

%%%%%%%%%%%%%%%%%%%%%%%%%%%%%%%%%%%%%%%%%%%%%%%%%%%%%%%%%%%%%%%%%%%%%%%%%%%%%%%%%%%%%%%%%%%%%%%%%%%%
%%%%%%%%%%%%%%%%%%%%%%%%%%%%%%%%%%%%%%%%%%%%%%%%%%%%%%%%%%%%%%%%%%%%%%%%%%%%%%%%%%%%%%%%%%%%%%%%%%%%

\newpage

\begin{QuestionFrame}
    Consider an arbitrary load (potentially both resistive and reactive) where:
    \begin{gather*}
        v(t) = V_m \cos{(\omega t + \theta_v)} \\
        i(t) = I_m \cos{(\omega t + \theta_i)}
    \end{gather*}
    
    Instantaneous power is given by:
    \begin{equation*}
        p(t) = v(t)\ i(t) = V_m I_m \cos{(\omega t + \theta_v)} \cos{(\omega t + \theta_i)}
    \end{equation*}

    Find the average power delivered to this load.
\end{QuestionFrame}

We can apply the following product-to-sum identity:
\begin{gather*}
    \boxed{
        2 \cos{A} \cos{B} = \cos{(A+B)} + \cos{(A-B)}
    } \\
    p(t) = \frac{V_m I_m}{2} \brackets*{\cos{(2 \omega t + \theta_v + \theta_i)} + \cos{(\theta_v - \theta_i)}}
\end{gather*}

To make things simpler, let's define the power angle $\theta$:
\begin{equation*}
    \theta := \theta_v - \theta_i
    \qquad \Longrightarrow \qquad
    \theta_i = \theta_v - \theta
\end{equation*}

Substituting into $p(t)$ then applying the following compound angle identity:
\begin{gather*}
    \boxed{
        \cos{(\alpha - \beta)} = \cos{\alpha} \cos{\beta} + \sin{\alpha} \sin{\beta}
    } \\
    \begin{aligned}
        p(t)
            &= \frac{V_m I_m}{2} \brackets*{\cos{\parens*{[2 \omega t + 2 \theta_v] - \theta}} + \cos{(\theta)}} \\
            &= \frac{V_m I_m}{2} \brackets*{\cos{(2 \omega t + 2 \theta_v)} \cos{(\theta)} + \sin{(2 \omega t + 2 \theta_v)} \sin{(\theta)} + \cos{(\theta)}} \\
            &= \frac{V_m I_m}{2} \cos{(\theta)} \brackets*{\underbrace{\cos{(2 \omega t + 2 \theta_v)}}_{\text{average value} = 0} + 1} + \frac{V_m I_m}{2} \sin{(\theta)} \underbrace{\sin{(2 \omega t + 2 \theta_v)}}_{\text{average value} = 0}
    \end{aligned}
\end{gather*}

To find average power, we notice that two time-dependent sinusoids have average values of zero, therefore our average power $P$ is:
\begin{equation*}
    P = \frac{V_m I_m}{2} \cos{(\theta)}
\end{equation*}



%%%%%%%%%%%%%%%%%%%%%%%%%%%%%%%%%%%%%%%%%%%%%%%%%%%%%%%%%%%%%%%%%%%%%%%%%%%%%%%%%%%%%%%%%%%%%%%%%%%%
%%%%%%%%%%%%%%%%%%%%%%%%%%%%%%%%%%%%%%%%%%%%%%%%%%%%%%%%%%%%%%%%%%%%%%%%%%%%%%%%%%%%%%%%%%%%%%%%%%%%

\newpage

\begin{QuestionFrame}
    For the circuit below, find average power and reactive power delivered to the entire load $\mathbf{Z} = R + jX$, $R$ alone, and $jX$ alone:
    \begin{itemize}
        \item in terms of voltage, $R$, and $X$, but NOT current, and
        \item in terms of current, $R$, and $X$, but NOT voltage.
    \end{itemize}
    \begin{figure}[H]\centering
    \begin{circuitikz}
        \draw
            (0,0)
            -- ++(0,1)
            to[V, invert, l=$\mathbf{V}$] ++(0,2)
            -- ++(0,1)
            to[short, i=$\mathbf{I}$] ++(3,0)
            to[R, l=$R$, v=$\mathbf{V}_R$] ++(0,-2)
            to[generic, l=$jX$, v=$\mathbf{V}_X$] ++(0,-2)
            -- (0,0)
        ;
    \end{circuitikz}
    %\caption{caption}
    %\label{fig:labelname}
    \end{figure}
\end{QuestionFrame}

Before we begin, we can define the following useful result:
\begin{equation*}
    \frac{1}{\cos{\theta} + j \sin{\theta}}
        = \frac{1}{\cos{\theta} + j \sin{\theta}} \frac{\cos{\theta} - j \sin{\theta}}{\cos{\theta} - j \sin{\theta}}
        = \frac{\cos{\theta} - j \sin{\theta}}{\cos^2{\theta} + \sin^2{\theta}}
        = \cos{\theta} - j \sin{\theta}
        = \cos{(-\theta)} + j \sin{(-\theta)}
\end{equation*}

Start by defining the voltage as the reference:
\begin{equation*}
    \mathbf{V} := V_m \phase{\ang{0}}
\end{equation*}

Thus, we can find the current:
\begin{equation*}
    \mathbf{I}
        = \frac{\mathbf{V}}{\mathbf{Z}}
        = \frac{V_m \phase{\ang{0}}}{\abs*{\mathbf{Z}} \phase{\theta}}
        = \frac{V_m}{\abs*{\mathbf{Z}}} \frac{1}{\cos{\theta} + j \sin{\theta}}
        %= \frac{V_m}{\abs*{\mathbf{Z}}} \frac{1}{\cos{\theta} + j \sin{\theta}} \frac{\cos{\theta} - j \sin{\theta}}{\cos{\theta} - j \sin{\theta}}
        %= \frac{V_m}{\abs*{\mathbf{Z}}} \frac{\cos{\theta} - j \sin{\theta}}{\cos^2{\theta} + \sin^2{\theta}}
        %= \frac{V_m}{\abs*{\mathbf{Z}}} \frac{\cos{\theta} - j \sin{\theta}}{1}
        = \frac{V_m}{\abs*{\mathbf{Z}}} \parens*{\cos{(-\theta)} + j \sin{(-\theta)}}
        := I_m \phase{-\theta}
\end{equation*}

From this, we get:
\begin{gather*}
    \mathbf{V}
        = \mathbf{I} \mathbf{Z}
        \\
    V_m
        = \parens*{R + jX} I_m \parens*{\cos{\theta} - j \sin{\theta}}
        \\
    V_m \frac{1}{\cos{\theta} - j \sin{\theta}}
        = I_m \parens*{R + jX}
        \\
    V_m \parens*{\cos{\theta} + j \sin{\theta}}
        = I_m \parens*{R + jX}
\end{gather*}

And similarly:
\begin{gather*}
    \mathbf{I}
        = \frac{\mathbf{V}}{\mathbf{Z}}
        \\
    I_m \parens*{\cos{\theta} - j \sin{\theta}}
        = \frac{V_m}{R + jX}
        = \frac{V_m}{R + jX} \frac{R - jX}{R - jX}
        = \frac{V_m \parens*{R - jX}}{R^2 + X^2}
\end{gather*}

Thus, we get:
\begin{align*}
    V_m \cos{\theta} &= I_m R \\
    V_m \sin{\theta} &= I_m X \\
    I_m \cos{\theta} &= V_m \frac{R}{R^2 + X^2} \\
    I_m \sin{\theta} &= V_m \frac{X}{R^2 + X^2}
\end{align*}

Using the equations for average power and reactive power (noting that the angle $\theta$ works because it is the angle in which current lags behind the voltage), we get our expressions for average and reactive power delivered to the entire load $\mathbf{Z}$:
\begin{equation*}
    \boxed{
        P = \frac{1}{2} V_m I_m \cos{\theta}
    }
    \qquad
    \boxed{
        Q = \frac{1}{2} V_m I_m \sin{\theta}
    }
\end{equation*}
\begin{subequations}
\begin{alignat}{3}
    P_{\mathbf{Z}} &= \frac{1}{2} I_m \parens*{I_m R} &
        \qquad & \Longrightarrow \qquad &
        P_{\mathbf{Z}} &= \frac{1}{2} I_m^2 R = I_{\text{rms}}^2 R
        \nonumber %\label{eq:z-avg-power-as-i-r-x}
        \\
    Q_{\mathbf{Z}} &= \frac{1}{2} I_m \parens*{I_m X} &
        \qquad & \Longrightarrow \qquad &
        Q_{\mathbf{Z}} &= \frac{1}{2} I_m^2 X = I_{\text{rms}}^2 X
        \nonumber %\label{eq:z-rtv-power-as-i-r-x}
        \\
    P_{\mathbf{Z}} &= \frac{1}{2} V_m \parens*{V_m \frac{R}{R^2 + X^2}} &
        \qquad & \Longrightarrow \qquad &
        P_{\mathbf{Z}} &= \frac{1}{2} V_m^2 \frac{R}{R^2 + X^2} = V_{\text{rms}}^2 \frac{R}{R^2 + X^2}
        \label{eq:z-avg-power-as-v-r-x}
        \\
    Q_{\mathbf{Z}} &= \frac{1}{2} V_m \parens*{V_m \frac{X}{R^2 + X^2}} &
        \qquad & \Longrightarrow \qquad &
        Q_{\mathbf{Z}} &= \frac{1}{2} V_m^2 \frac{X}{R^2 + X^2} = V_{\text{rms}}^2 \frac{X}{R^2 + X^2}
        \label{eq:z-rtv-power-as-v-r-x}
\end{alignat}
\end{subequations}

Now, to find the average and reactive power delivered to $R$ and $jX$, we first note that $R$ will only receive average power while $jX$ will only receive reactive power:
\begin{subequations}
\begin{align*}
    P_X &= 0 \\
    Q_R &= 0
\end{align*}
\end{subequations}

Due to a common current $\mathbf{I}$ and conservation of AC power, we can also deduce:
\begin{subequations}
\begin{align*}
    P_R &= P_{\mathbf{Z}} = \frac{1}{2} I_m^2 R = I_{\text{rms}}^2 R \\
    Q_X &= Q_{\mathbf{Z}} = \frac{1}{2} I_m^2 X = I_{\text{rms}}^2 X
\end{align*}
\end{subequations}

However, $R$ and $jX$ experience different voltages $\mathbf{V}_R := V_{Rm} \phase{\theta_R}$ and $\mathbf{V}_X := V_{Xm} \phase{\theta_X}$.

By voltage division:
\begin{align*}
    \mathbf{V}_R
        &= \frac{R}{R + jX} \mathbf{V}
        = V_m \frac{R}{\mathbf{Z}}
        = V_m \frac{R}{\abs*{\mathbf{Z}} \phase{\theta}}
        = V_m \frac{R}{\sqrt{R^2 + X^2} \parens*{\cos{\theta} + j \sin{\theta}}}
        = V_m \frac{R}{\sqrt{R^2 + X^2}} \parens*{\cos{(-\theta)} + j \sin{(-\theta)}}
        \\
    \mathbf{V}_X
        &= \frac{jX}{R + jX} \mathbf{V}
        = V_m \frac{jX}{\mathbf{Z}}
        = V_m \frac{jX}{\abs*{\mathbf{Z}} \phase{\theta}}
        = V_m \frac{jX}{\sqrt{R^2 + X^2} \parens*{\cos{\theta} + j \sin{\theta}}}
        = V_m \frac{X}{\sqrt{R^2 + X^2}} j \parens*{\cos{(-\theta)} + j \sin{(-\theta)}}
        \\
\end{align*}

We can find the magnitudes, rearrange, then substitute into equations \eqref{eq:z-avg-power-as-v-r-x} and \eqref{eq:z-rtv-power-as-v-r-x}:
\begin{alignat*}{5}
    V_{Rm} &= V_m \frac{R}{\sqrt{R^2 + X^2}} &
        \qquad & \Longrightarrow \qquad &
        \frac{V_{Rm}^2}{R} &= V_m^2 \frac{R}{R^2 + X^2}
        \qquad & \Longrightarrow \qquad &
        P_R &= P_{\mathbf{Z}} = \frac{1}{2} V_m^2 \frac{R}{R^2 + X^2} = \frac{1}{2} \frac{V_{Rm}^2}{R} = \frac{V_{Rrms}^2}{R}
        \\
    V_{Xm} &= V_m \frac{X}{\sqrt{R^2 + X^2}} &
        \qquad & \Longrightarrow \qquad &
        \frac{V_{Xm}^2}{X} &= V_m^2 \frac{X}{R^2 + X^2}
        \qquad & \Longrightarrow \qquad &
        Q_X &= Q_{\mathbf{Z}} = \frac{1}{2} V_m^2 \frac{X}{R^2 + X^2} = \frac{1}{2} \frac{V_{Xm}^2}{X} = \frac{V_{Xrms}^2}{X}
\end{alignat*}

Thus, we have found our solutions.


%%%%%%%%%%%%%%%%%%%%%%%%%%%%%%%%%%%%%%%%%%%%%%%%%%%%%%%%%%%%%%%%%%%%%%%%%%%%%%%%%%%%%%%%%%%%%%%%%%%%
%%%%%%%%%%%%%%%%%%%%%%%%%%%%%%%%%%%%%%%%%%%%%%%%%%%%%%%%%%%%%%%%%%%%%%%%%%%%%%%%%%%%%%%%%%%%%%%%%%%%

\newpage

\begin{QuestionFrame}
    Solve the ODE:
    \begin{equation*}
        -y \,\diff{y} + \parens*{x + \sqrt{xy}} \,\diff{y} = 0
    \end{equation*}
\end{QuestionFrame}

Rearranging the ODE:
\begin{equation*}
    \frac{\diff{x}}{\diff{y}} = \frac{x + \sqrt{xy}}{y} = \frac{x}{y} + \frac{\sqrt{x}}{\sqrt{y}}
\end{equation*}

From this form, we can try the following substitution:
\begin{equation*}
    z := \frac{x}{y}
    \qquad \Longrightarrow \qquad
    x(y) = z(y) y
    \qquad \Longrightarrow \qquad
    \frac{\diff{x}}{\diff{y}} = z \xcancelto{1}{\frac{\diff{y}}{\diff{y}}} + y \frac{\diff{z}}{\diff{y}} = z + y \frac{\diff{z}}{\diff{y}}
\end{equation*}

Substituting into our ODE, rearranging, then integrating both sides:
\begin{gather*}
    z + y \frac{\diff{z}}{\diff{y}} = \frac{zy + \sqrt{z y^2}}{y} = \frac{z \cancel{y} + \cancel{y} \sqrt{z}}{\cancel{y}} = z + \sqrt{z} \\
    y \frac{\diff{z}}{\diff{y}} = \sqrt{z} \\
    z^{-\frac{1}{2}} \,\diff{z} = \frac{1}{y} \,\diff{y} \\
    2 \sqrt{z} = \ln{\abs*{y}} + C
\end{gather*}

Reverting the substitution, we get our solution to the ODE:
\begin{equation*}
    2 \frac{\sqrt{x}}{\sqrt{y}} = \ln{\abs*{y}} + C, \qquad C \in \mathbb{R}
\end{equation*}

\bigskip

\textit{OPTIONAL: We shall now verify our solution.}

Rearranging our solution:
\begin{gather*}
    \sqrt{x} = \frac{1}{2} \sqrt{y} \parens*{\ln{\abs*{y}} + C}
    \qquad \Longrightarrow \qquad
    x = \frac{1}{4} y \parens*{\ln{\abs*{y}} + C}^2
    \\
    \begin{aligned}
        4 \frac{\diff{x}}{\diff{y}}
            &= y \frac{\diff{}}{\diff{y}}\parens*{\parens*{\ln{\abs*{y}} + C}^2} + \parens*{\ln{\abs*{y}} + C}^2 \xcancelto{1}{\frac{\diff{y}}{\diff{y}}} \\
            &= 2 y \parens*{\ln{\abs*{y}} + C} \frac{1}{y} + \parens*{\ln{\abs*{y}} + C}^2
    \end{aligned}
\end{gather*}

Thus, LHS of our ODE is:
\begin{equation*}
    \lhs = \frac{1}{2} \parens*{\ln{\abs*{y}} + C} + \frac{1}{4} \parens*{\ln{\abs*{y}} + C}^2
\end{equation*}

For the RHS, we substitute our solution:
\begin{align*}
    \rhs
        &= \frac{\frac{1}{4} \cancel{y} \parens*{\ln{\abs*{y}} + C}^2}{\cancel{y}} + \sqrt{\frac{\frac{1}{4} \cancel{y} \parens*{\ln{\abs*{y}} + C}^2}{\cancel{y}}}
        = \frac{1}{4} \parens*{\ln{\abs*{y}} + C}^2 + \sqrt{\frac{1}{4} \parens*{\ln{\abs*{y}} + C}^2} \\
        &= \frac{1}{4} \parens*{\ln{\abs*{y}} + C}^2 + \frac{1}{2} \parens*{\ln{\abs*{y}} + C} = \lhs
\end{align*}

Thus, our solution has been verified.


%%%%%%%%%%%%%%%%%%%%%%%%%%%%%%%%%%%%%%%%%%%%%%%%%%%%%%%%%%%%%%%%%%%%%%%%%%%%%%%%%%%%%%%%%%%%%%%%%%%%
%%%%%%%%%%%%%%%%%%%%%%%%%%%%%%%%%%%%%%%%%%%%%%%%%%%%%%%%%%%%%%%%%%%%%%%%%%%%%%%%%%%%%%%%%%%%%%%%%%%%

\newpage

\begin{QuestionFrame}
    new question!
\end{QuestionFrame}

\end{document}
