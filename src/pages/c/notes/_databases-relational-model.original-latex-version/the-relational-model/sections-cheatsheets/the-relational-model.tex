%\subsection{Databases: The Relational Model and Relational Algebra}%
%\label{sub:relational-algebra}

\begin{multicols}{2}

    % TODO: Find a way to work this into this cheatsheet.
    %\begin{CheatsheetEntryFrameNew}

    %    \CheatsheetEntryTitle{Relations}

    %    The term \textit{relation} can refer to either:
    %    \begin{itemize}
    %        \item a \textit{relation schema}, or
    %        \item a \textit{relation instance}.
    %    \end{itemize}

    %    Intended usage should be clear from context.

    %\end{CheatsheetEntryFrameNew}

    \begin{CheatsheetEntryFrame}

        \CheatsheetEntryTitle{Relation Schemas}

        Relation schemas are \myul{finite sets} of \textit{attributes}.

        Notation for some relation $R$ with attributes $A, B, C, D$ and primary key $B, D$:
        \begin{equation*}
            R(A, \underline{B}, C, \underline{D})
        \end{equation*}
        %% I want a visually annotated version like this below.
        %% Problem is that the relation name annotation looks really awkward.
        %\newcommand{\X}{\rule{0mm}{1.1em}}
        %\begin{equation*}
        %    {
        %        \color{mycontrastred}
        %        \overbrace{\color{black} \X \hspace{1em} R}^{\mathclap{\substack{\textbf{\textit{relation}}\\\textbf{\textit{name}}}}}
        %    }
        %    ({
        %        \color {mycontrastpurple}
        %        \overbrace{\color{black} \X A, B, C, \dots}^{\textbf{\textit{attributes}}}
        %    })
        %\end{equation*}

        The \textit{domain} of a schema is the set of all possible tuples that can be associated with the schema.

        For relation $R$, this is denoted:
        \begin{equation*}
            \dom{(R)}
        \end{equation*}

        %\CheatsheetEntryExtraSeparation

        \CheatsheetEntryTitle{Keys}

        A \textit{\myul{superkey}} of a relation schema is any subset of that schema that can uniquely identify a tuple.

        In other words, all tuples have unique values in the superkey.

        Or more formally, for $r(R)$ and any two distinct tuples $t_1, t_2 \in r$, if $t_1[K] \neq t_2[K]$ for some $K \subseteq R$, then $K$ is a superkey.

        Note that for $r(R)$, $R$ is always a superkey.

        \vspace{\TextExtraSkip}%
        A \textit{\myul{candidate key}} is a ``minimal superkey".

        More formally, superkey $K$ is a candidate key if no subset $K' \subset K$ exists that is also a superkey.

        \vspace{\TextExtraSkip}%
        A \textit{\myul{primary key}} is the specific candidate key designated to a relation schema.

        All relations must have exactly one primary key.

        \vspace{\TextExtraSkip}%
        A \textit{\myul{foreign key}} of a relation schema is a subset of attributes that reference a particular primary key.

        The referenced primary key is often a different relation, but may also be of the same relation.

        %% I initially started writing this, but realized that null values are not part of the core relational model.
        %Constraints on primary key:
        %\begin{itemize}
        %    \item \textit{Uniqueness constraint}: Primary key attributes must never allow duplicate values.
        %    \item \textit{Entity integrity:} Primary key must never be null.
        %\end{itemize}
        

    \end{CheatsheetEntryFrame}

    \begin{CheatsheetEntryFrame}

        \CheatsheetEntryTitle{Attributes}

        Attributes are labels for parts of a schema.

        The \textit{domain} of an attribute is the set of values that can be associated with the attribute.

        For attribute $A$, this is denoted:
        \begin{equation*}
            \dom{(A)}
        \end{equation*}

    \end{CheatsheetEntryFrame}

    %\Todo{Discuss the domain of the whole schema?}

    \MulticolsBreak

    \begin{CheatsheetEntryFrame}

        \CheatsheetEntryTitle{Relation Instances}

        Relation instances are \myul{finite sets} of \textit{tuples}.

        To show that $r$ is an instance of schema $R$, we write:
        \begin{equation*}
            r(R)
        \end{equation*}

        %Tuples are mappings from schema $R$ to domain $\dom{(R)}$.

        \SubsectionFrameRemoveSeparation
        \begin{RelAlgSubsection}{Example}
        \begin{center}
            \newcommand{\Y}{0.44}
            \newcommand{\Yhalf}{0.22}
            \newcommand{\MyTableCell}[3]{%
                (R#1 -| C#2) node[right, align=left, font=\footnotesize] {\vphantom{$M_I^{I^x}$}\texttt{#3}}%
            }
            \begin{tikzpicture}[scale=1, transform shape]
                \path
                    (0,0)      coordinate (LeftEdge)
                    ++(0.08,0) coordinate (C1)
                    ++(1.98,0) coordinate (C2)
                    ++(1.50,0) coordinate (C3)
                    ++(2.10,0) coordinate (RightEdge)
                    ++(0.20,0) coordinate (RightOS1)
                    ++(0.50,0) coordinate (RightOS2)
                ;
                \begin{scope}[shift={(0,0)}]
                    \path
                        (0,0)         coordinate (Top)
                        ++(0,-\Yhalf) coordinate (RA)
                        ++(0,-\Yhalf) coordinate (Bottom)

                        (Top)
                        ++(0, 0.040) % Slight extra offset
                        ++(0, \Yhalf) coordinate (RT) % Title row

                        (Top)
                        ++(0,-0.1) coordinate (TopOS1)

                        (Top)
                        ++(0,1.2) coordinate (TopOS2)
                        ++(0,0.5) coordinate (TopOS3)
                    ;
                    \draw
                        (Top -| LeftEdge) rectangle (RightEdge |- Bottom)
                    ;
                    \draw
                        \MyTableCell{T}{1}{\textbf{Customers}}

                        \MyTableCell{A}{1}{customerid}
                        \MyTableCell{A}{2}{name}
                        \MyTableCell{A}{3}{address}
                    ;
                \end{scope}
                \begin{scope}[shift={(0,-0.52)}]
                    \path
                        (0,0)         coordinate (Top)
                        ++(0,-\Yhalf) coordinate (R1)
                        ++(0,-\Y)     coordinate (R2)
                        ++(0,-\Y)     coordinate (R3) coordinate (HorizontalMid)
                        ++(0,-\Y)     coordinate (R4)
                        ++(0,-\Y)     coordinate (R5)
                        ++(0,-\Yhalf) coordinate (Bottom)
                    ;
                    \draw
                        (Top -| LeftEdge) rectangle (RightEdge |- Bottom)
                    ;
                    \draw
                        \MyTableCell{1}{1}{1774504}
                        \MyTableCell{1}{2}{Amy}
                        \MyTableCell{1}{3}{Surry Hills}

                        \MyTableCell{2}{1}{4389167}
                        \MyTableCell{2}{2}{Chris}
                        \MyTableCell{2}{3}{Richmond}

                        \MyTableCell{3}{1}{4622780}
                        \MyTableCell{3}{2}{Josh}
                        \MyTableCell{3}{3}{North Ryde}
                        
                        \MyTableCell{4}{1}{5691729}
                        \MyTableCell{4}{2}{Sam}
                        \MyTableCell{4}{3}{Kensington}

                        \MyTableCell{5}{1}{9527291}
                        \MyTableCell{5}{2}{Vanessa}
                        \MyTableCell{5}{3}{Richmond}
                    ;

                    \draw[mycontrastblue]
                        (HorizontalMid -| RightOS2)
                        ++(0.10,0) coordinate (RowArrowConvg)
                        ++(0.05,0) node[right, align=left, font=\small] {
                            \textbf{\textit{tuples,}} \\
                            \textbf{\textit{rows,}} \\
                            \textbf{\textit{records}}
                        }
                    ;
                    \draw[-stealth, cap=round, line width=2.0pt, mycontrastblue] (R1 -| RightOS2) -- (R1 -| RightOS1);
                    \draw[-stealth, cap=round, line width=2.0pt, mycontrastblue] (R2 -| RightOS2) -- (R2 -| RightOS1);
                    \draw[-stealth, cap=round, line width=2.0pt, mycontrastblue] (R3 -| RightOS2) -- (R3 -| RightOS1);
                    \draw[-stealth, cap=round, line width=2.0pt, mycontrastblue] (R4 -| RightOS2) -- (R4 -| RightOS1);
                    \draw[-stealth, cap=round, line width=2.0pt, mycontrastblue] (R5 -| RightOS2) -- (R5 -| RightOS1);
                    \draw[          cap=round, line width=2.0pt, mycontrastblue] (R1 -| RightOS2) -- (R5 -| RightOS2);
                    %\draw[         cap=round, line width=2.0pt, mycontrastblue] (RowArrowConvg) -- (RowArrowConvg -| RightOS2);

                    \draw[mycontrastpurple]
                        (TopOS2 -| C3)
                        ++(1.8, 0) coordinate (ColArrowConvgRef)
                        ++(0.05, 0) node[right, align=left, font=\small] {
                            \textbf{\textit{attributes,}} \\
                            \textbf{\textit{columns,}} \\
                            \textbf{\textit{fields}}
                        }
                    ;
                    \path
                        (RA -| C1) ++(1.75, 0.30) coordinate (ColArrowEnd1)
                        (RA -| C2) ++(0.85, 0.30) coordinate (ColArrowEnd2)
                        (RA -| C3) ++(0.80, 0.30) coordinate (ColArrowEnd3)
                        (ColArrowConvgRef) ++(0, -0.1) coordinate (ColArrowConvg)
                    ;
                    \draw[-stealth, cap=round, line width=2.0pt, mycontrastpurple] (ColArrowConvg) -- (ColArrowEnd1);
                    \draw[-stealth, cap=round, line width=2.0pt, mycontrastpurple] (ColArrowConvg) -- (ColArrowEnd2);
                    \draw[-stealth, cap=round, line width=2.0pt, mycontrastpurple] (ColArrowConvg) -- (ColArrowEnd3);

                    \draw[mycontrastred]
                        (TopOS3 -| C2)
                        ++(0.30, 0) coordinate (RelArrowConvgRef)
                        ++(0.05, 0) node[right, align=left, font=\small] {
                            \textbf{\textit{relation,}} \\
                            \textbf{\textit{table}}
                        }
                    ;
                    \path
                        (RT -| C1) ++(1.00, 0.25) coordinate (RelArrowEnd)
                        (RelArrowConvgRef) ++(0, -0.3) coordinate (RelArrowConvg)
                    ;
                    \draw[-stealth, cap=round, line width=2.0pt, mycontrastred] (RelArrowConvg) -- (RelArrowEnd);

                    % An attempt was made to make smooth lines...
                    %\draw[-stealth, cap=round, line width=1.5pt]
                    %    plot [smooth] coordinates { (RowArrowConvg) (R1 -| RightOS2) (R1 -| RightOS1) };
                    %\draw[-stealth, cap=round, line width=1.5pt]
                    %    plot [smooth] coordinates { (RowArrowConvg) (R2 -| RightOS2) (R2 -| RightOS1) };
                    %\draw[-stealth, cap=round, line width=1.5pt]
                    %    plot [smooth] coordinates { (RowArrowConvg) (R3 -| RightOS2) (R3 -| RightOS1) };
                    %\draw[-stealth, cap=round, line width=1.5pt]
                    %    plot [smooth] coordinates { (RowArrowConvg) (R4 -| RightOS2) (R4 -| RightOS1) };
                    %\draw[-stealth, cap=round, line width=1.5pt]
                    %    plot [smooth] coordinates { (RowArrowConvg) (R5 -| RightOS2) (R5 -| RightOS1) };
                \end{scope}
            \end{tikzpicture}

            %%
            %% THE OTHER EXAMPLE RELATIONS BELOW ARE FOR A MORE EXTENDED EXAMPLE.
            %% I AM NOT INCLUDING THEM FOR SPACE REASONS... FOR NOW.
            %%

            %\medskip

            %{\footnotesize%
            %\begin{tabular}{|llr|}
            %    \multicolumn{3}{l}{\ttd{Accounts}}
            %        \\ \hline
            %    \multicolumn{1}{|l}{\ttd{branch}}
            %        & \multicolumn{1}{l}{\ttd{accountno}}
            %        & \multicolumn{1}{l|}{\ttd{balance}}
            %        \\ \hline\hline
            %    \ttd{Richmond}
            %        & \ttd{A-02772}
            %        & \ttd{20.87}
            %        \\
            %    \ttd{Macquarie} % The actual suburb is called Macquarie Park, but I want a smaller table.
            %        & \ttd{J-31553}
            %        & \ttd{60899.58}
            %        \\
            %    \ttd{Richmond}
            %        & \ttd{W-40018}
            %        & \ttd{84731.08}
            %        \\
            %    \ttd{Haymarket}
            %        & \ttd{A-74884}
            %        & \ttd{483.94}
            %        \\
            %    \ttd{Haymarket}
            %        & \ttd{P-85953}
            %        & \ttd{7294.62}
            %        \\ \hline
            %\end{tabular}%

            %\medskip
            %%\hspace{2.0ex} % TODO: Prefer to instead make these two tables side-by-side if possible.

            %\begin{tabular}{|ll|}
            %    \multicolumn{2}{l}{\ttd{HeldBy}}
            %        \\ \hline
            %    \multicolumn{1}{|l}{\ttd{accountno}}
            %        & \multicolumn{1}{l|}{\ttd{owner}}
            %        \\ \hline\hline
            %    \ttd{A-02772}
            %        & \ttd{4389167}
            %        \\
            %    \ttd{A-02772}
            %        & \ttd{9527291}
            %        \\
            %    \ttd{J-31553}
            %        & \ttd{4622780}
            %        \\
            %    \ttd{W-40018}
            %        & \ttd{4389167}
            %        \\
            %    \ttd{W-40018}
            %        & \ttd{9527291}
            %        \\
            %    \ttd{A-74884}
            %        & \ttd{1774504}
            %        \\
            %    \ttd{P-85953}
            %        & \ttd{1774504}
            %        \\ \hline
            %\end{tabular}%
            %}
            
        \end{center}
        \end{RelAlgSubsection}

    \end{CheatsheetEntryFrame}

    \begin{CheatsheetEntryFrame}

        \CheatsheetEntryTitle{Tuples}

        Tuples are \myul{mappings} from schema to schema domain.

        Concretely, a tuple can be thought of as a set of attribute-value pairs.

        We will use the following notation in this cheatsheet:
        \begin{psmallindent}
            $t[X]$ extracts attribute set $X$. \\[0mm]
            $t[X]$ is itself a tuple.

            $t[A]$ extracts a single attribute $A$. \\[0mm]
            $t[A]$ is a value. \Exn{\textit{(Yes, I know this is confusing.)}}

            $(t_r : t_s)$ combines tuples $t_r$ and $t_s$. \\[0mm]
            $(t_r : t_s)$ is itself a tuple.
        \end{psmallindent}

    \end{CheatsheetEntryFrame}

    \begin{CheatsheetEntryFrame}

        \CheatsheetEntryTitle{The Null Value} \MarkExtendedRelAlg

        The null value is a special value that may be part of an attribute domain.

        We will use the symbol $\relnullvalue$ to represent null in this cheatsheet.

        While the null value is an important part of the relational model, \myul{it is not a core part of relational algebra}.

    \end{CheatsheetEntryFrame}

    %\begin{CheatsheetEntryFrame}

    %    \CheatsheetEntryTitle{Constraints} {\scriptsize \textsc{(Brief Overview)}}

    %    TODO: Consider doing a section on this!

    %\end{CheatsheetEntryFrame}

\end{multicols}
\newpage
\begin{multicols}{2}
    
    \begin{CheatsheetEntryFrame}

        \CheatsheetEntryTitle{Selection}

        Produces the subset of tuples that satisfy a condition.
        
        For $r(R)$ and selection condition $c$:
        \begin{equation*}
            \relselect_{c}{(r)} = \setdef{t \in r}{c(t)}
        \end{equation*}

        \SubsectionFrameRemoveSeparation
        \begin{RelAlgSubsection}{Example}
        \begin{center}
            {\footnotesize%
                \begin{tabular}{|ccc|}
                    \multicolumn{3}{l}{\normalsize $r$}
                        \\ \hline
                    \multicolumn{1}{|c}{$A$}
                        & \multicolumn{1}{c}{$B$}
                        & \multicolumn{1}{c|}{$C$}
                        \\ \hline\hline
                    $a_1$ & $2$ & $x$ \\
                    $a_2$ & $8$ & $x$ \\
                    $a_3$ & $7$ & $x$ \\
                    $a_4$ & $9$ & $y$ \\ \hline
                \end{tabular}
                \qquad \qquad
                \begin{tabular}{|ccc|c}
                    \multicolumn{4}{l}{\normalsize $\relselect_{(B>5) \wedge (C=x)}{(r)}$}
                        \\ \cline{1-3}
                    \multicolumn{1}{|c}{$A$}
                        & \multicolumn{1}{c}{$B$}
                        & \multicolumn{1}{c|}{$C$}
                        & \multicolumn{1}{c}{} % Empty cell
                        \\ \cline{1-3} \cline{1-3}
                    $a_2$ & $8$ & $x$ & \multicolumn{1}{c}{} \\
                    $a_3$ & $7$ & $x$ & \multicolumn{1}{c}{} \\ \cline{1-3}
                \end{tabular}
            }
        \end{center}
        \end{RelAlgSubsection}

    \end{CheatsheetEntryFrame}

    \begin{CheatsheetEntryFrame}

        \CheatsheetEntryTitle{Projection}

        Produces a relation with a subset of attributes.

        For $r(R)$ and attribute set $X$:
        \begin{equation*}
            \relproject_{X}{(r)} = \setdef{t[X]}{t \in r}
        \end{equation*}

        \SubsectionFrameRemoveSeparation
        \begin{RelAlgSubsection}{Example}
        \begin{center}
            {\footnotesize%
                \begin{tabular}{|cccc|}
                    \multicolumn{4}{l}{\normalsize $r$}
                        \\ \hline
                    \multicolumn{1}{|c}{$A$}
                        & \multicolumn{1}{c}{$B$}
                        & \multicolumn{1}{c}{$C$}
                        & \multicolumn{1}{c|}{$D$}
                        \\ \hline\hline
                    $a_1$ & $b_1$ & $c_1$ & $d_1$ \\
                    $a_2$ & $b_2$ & $c_2$ & $d_2$ \\
                    $a_3$ & $b_3$ & $c_3$ & $d_3$ \\ \hline
                \end{tabular}
                \qquad \qquad
                \begin{tabular}{|cc|}
                    \multicolumn{2}{l}{\normalsize $\relproject_{B, D}{(r)}$}
                        \\ \hline
                    \multicolumn{1}{|c}{$B$}
                        & \multicolumn{1}{c|}{$D$}
                        \\ \hline\hline
                    $b_1$ & $d_1$ \\
                    $b_2$ & $d_2$ \\
                    $b_3$ & $d_3$ \\ \hline
                \end{tabular}
            }
        \end{center}
        \end{RelAlgSubsection}

    \end{CheatsheetEntryFrame}

    \begin{CheatsheetEntryFrame}

        \CheatsheetEntryTitle{Rename}

        Casts a relation to a different schema.

        For $r(R)$ and a compatible schema $S$:
        \begin{equation*}
            \relrename_{S}{(r)}
        \end{equation*}

        \SubsectionFrameRemoveSeparation
        \begin{RelAlgSubsection}{Example}
        \begin{center}
            {\footnotesize%
                \begin{tabular}{|ccc|}
                    \multicolumn{3}{l}{\normalsize $r$}
                        \\ \hline
                    \multicolumn{1}{|c}{$A$}
                        & \multicolumn{1}{c}{$B$}
                        & \multicolumn{1}{c|}{$C$}
                        \\ \hline\hline
                    $a_1$ & $b_1$ & $c_1$ \\
                    $a_2$ & $b_2$ & $c_2$ \\
                    $a_3$ & $b_3$ & $c_3$ \\ \hline
                \end{tabular}
                \qquad \qquad
                \begin{tabular}{|ccc|}
                    \multicolumn{3}{l}{\normalsize $\relrename_{S(D, E, F)}{(r)}$}
                        \\ \hline
                    \multicolumn{1}{|c}{$D$}
                        & \multicolumn{1}{c}{$E$}
                        & \multicolumn{1}{c|}{$F$}
                        \\ \hline\hline
                    $a_1$ & $b_1$ & $c_1$ \\
                    $a_2$ & $b_2$ & $c_2$ \\
                    $a_3$ & $b_3$ & $c_3$ \\ \hline
                \end{tabular}
            }
        \end{center}
        \end{RelAlgSubsection}

    \end{CheatsheetEntryFrame}

    \MulticolsBreak

    \begin{CheatsheetEntryFrame}

        \CheatsheetEntryTitle{Union, Intersection, and Difference}

        Set theoretic operations between \textit{union-compatible} relations (i.e. same schema).

        For $r_1(R)$ and $r_2(R)$:
        \begin{align*}
            r_1 \cup r_2 &= \setdef{t}{(t \in r_1) \vee   (t \in r_2)} \\[\EqExtraSkip]
            r_1 \cap r_2 &= \setdef{t}{(t \in r_1) \wedge (t \in r_2)} \\[\EqExtraSkip]
            r_1 - r_2    &= \setdef{t}{(t \in r_1) \wedge (t \notin r_2)}
        \end{align*}

        \SubsectionFrameRemoveSeparation
        \begin{RelAlgSubsection}{Example}
            \ThreeColumnsMinipages{%
                \footnotesize%
                \begin{tabular}{|ccc|}
                    \multicolumn{3}{l}{\normalsize $r_1$}
                        \\ \hline
                    \multicolumn{1}{|c}{$A$}
                        & \multicolumn{1}{c}{$B$}
                        & \multicolumn{1}{c|}{$C$}
                        \\ \hline\hline
                    $\memphRC{a_1}$ & $\memphRC{b_1}$ & $\memphRC{c_1}$ \\
                    $\memphRC{a_2}$ & $\memphRC{b_2}$ & $\memphRC{c_2}$ \\
                    $\memphRC{a_3}$ & $\memphRC{b_3}$ & $\memphRC{c_3}$ \\
                    $\memphPC{a_4}$ & $\memphPC{b_4}$ & $\memphPC{c_4}$ \\
                    $\memphPC{a_5}$ & $\memphPC{b_5}$ & $\memphPC{c_5}$ \\ \hline
                \end{tabular}

                \vspace{0.8ex}

                \begin{tabular}{|ccc|}
                    \multicolumn{3}{l}{\normalsize $r_2$}
                        \\ \hline
                    \multicolumn{1}{|c}{$A$}
                        & \multicolumn{1}{c}{$B$}
                        & \multicolumn{1}{c|}{$C$}
                        \\ \hline\hline
                    $\memphPC{a_4}$ & $\memphPC{b_4}$ & $\memphPC{c_4}$ \\
                    $\memphPC{a_5}$ & $\memphPC{b_5}$ & $\memphPC{c_5}$ \\
                    $\memphBC{a_6}$ & $\memphBC{b_6}$ & $\memphBC{c_6}$ \\
                    $\memphBC{a_7}$ & $\memphBC{b_7}$ & $\memphBC{c_7}$ \\ \hline
                \end{tabular}
            }{%
                \footnotesize%
                \begin{tabular}{|ccc|}
                    \multicolumn{3}{l}{\normalsize $r_1 \cup r_2$}
                        \\ \hline
                    \multicolumn{1}{|c}{$A$}
                        & \multicolumn{1}{c}{$B$}
                        & \multicolumn{1}{c|}{$C$}
                        \\ \hline\hline
                    $\memphRC{a_1}$ & $\memphRC{b_1}$ & $\memphRC{c_1}$ \\
                    $\memphRC{a_2}$ & $\memphRC{b_2}$ & $\memphRC{c_2}$ \\
                    $\memphRC{a_3}$ & $\memphRC{b_3}$ & $\memphRC{c_3}$ \\
                    $\memphPC{a_4}$ & $\memphPC{b_4}$ & $\memphPC{c_4}$ \\
                    $\memphPC{a_5}$ & $\memphPC{b_5}$ & $\memphPC{c_5}$ \\
                    $\memphBC{a_6}$ & $\memphBC{b_6}$ & $\memphBC{c_6}$ \\
                    $\memphBC{a_7}$ & $\memphBC{b_7}$ & $\memphBC{c_7}$ \\ \hline
                \end{tabular}
            }{%
                \footnotesize%
                \begin{tabular}{|ccc|}
                    \multicolumn{3}{l}{\normalsize $r_1 - r_2$}
                        \\ \hline
                    \multicolumn{1}{|c}{$A$}
                        & \multicolumn{1}{c}{$B$}
                        & \multicolumn{1}{c|}{$C$}
                        \\ \hline\hline
                    $\memphRC{a_1}$ & $\memphRC{b_1}$ & $\memphRC{c_1}$ \\
                    $\memphRC{a_2}$ & $\memphRC{b_2}$ & $\memphRC{c_2}$ \\
                    $\memphRC{a_3}$ & $\memphRC{b_3}$ & $\memphRC{c_3}$ \\ \hline
                \end{tabular}

                \vspace{0.8ex}

                \begin{tabular}{|ccc|}
                    \multicolumn{3}{l}{\normalsize $r_1 \cap r_2$}
                        \\ \hline
                    \multicolumn{1}{|c}{$A$}
                        & \multicolumn{1}{c}{$B$}
                        & \multicolumn{1}{c|}{$C$}
                        \\ \hline\hline
                    $\memphPC{a_4}$ & $\memphPC{b_4}$ & $\memphPC{c_4}$ \\
                    $\memphPC{a_5}$ & $\memphPC{b_5}$ & $\memphPC{c_5}$ \\ \hline
                \end{tabular}

                \vspace{0.8ex}

                \begin{tabular}{|ccc|}
                    \multicolumn{3}{l}{\normalsize $r_2 - r_1$}
                        \\ \hline
                    \multicolumn{1}{|c}{$A$}
                        & \multicolumn{1}{c}{$B$}
                        & \multicolumn{1}{c|}{$C$}
                        \\ \hline\hline
                    $\memphBC{a_6}$ & $\memphBC{b_6}$ & $\memphBC{c_6}$ \\
                    $\memphBC{a_7}$ & $\memphBC{b_7}$ & $\memphBC{c_7}$ \\ \hline
                \end{tabular}
            }
        \end{RelAlgSubsection}

    \end{CheatsheetEntryFrame}

    \begin{CheatsheetEntryFrame}

        \CheatsheetEntryTitle{Cartesian Product}

        Produces every possible combination of tuples from two relations.

        For $r(R)$ and $s(S)$:
        \begin{equation*}
            r \times s = \setdef{(t_r:t_s)}{(t_r \in r) \wedge (t_s \in s)}
        \end{equation*}

        \SubsectionFrameRemoveSeparation
        \begin{RelAlgSubsection}{Example}
        \begin{center}
            {\footnotesize%
                \begin{tabular}{|ccc|}
                    \multicolumn{3}{l}{\normalsize $s$}
                        \\ \cline{1-2}
                    \multicolumn{1}{|c}{$M$}
                        & \multicolumn{1}{c|}{$N$}
                        & \multicolumn{1}{c}{} % Empty cell
                        \\ \cline{1-2} \cline{1-2}
                    $m_1$ & \multicolumn{1}{c|}{$n_1$} & \multicolumn{1}{c}{} \\
                    $m_2$ & \multicolumn{1}{c|}{$n_2$} & \multicolumn{1}{c}{} \\
                    $m_3$ & \multicolumn{1}{c|}{$n_3$} & \multicolumn{1}{c}{} \\ \cline{1-2}
                    \multicolumn{3}{c}{} \\[1.5ex] % Empty Gap
                    \multicolumn{3}{l}{\normalsize $r$}
                        \\ \hline
                    \multicolumn{1}{|c}{$A$}
                        & \multicolumn{1}{c}{$B$}
                        & \multicolumn{1}{c|}{$C$}
                        \\ \hline\hline
                    $a_1$ & $b_1$ & $c_1$ \\
                    $a_2$ & $b_2$ & $c_2$ \\ \hline
                \end{tabular}
                \qquad \quad
                \begin{tabular}{|ccccc|}
                    \multicolumn{5}{l}{\normalsize $r \times s$}
                        \\ \hline
                    \multicolumn{1}{|c}{$A$}
                        & \multicolumn{1}{c}{$B$}
                        & \multicolumn{1}{c}{$C$}
                        & \multicolumn{1}{c}{$M$}
                        & \multicolumn{1}{c|}{$N$}
                        \\ \hline\hline
                    $a_1$ & $b_1$ & $c_1$ & $m_1$ & $n_1$ \\
                    $a_1$ & $b_1$ & $c_1$ & $m_2$ & $n_2$ \\
                    $a_1$ & $b_1$ & $c_1$ & $m_3$ & $n_3$ \\
                    $a_2$ & $b_2$ & $c_2$ & $m_1$ & $n_1$ \\
                    $a_2$ & $b_2$ & $c_2$ & $m_2$ & $n_2$ \\
                    $a_2$ & $b_2$ & $c_2$ & $m_3$ & $n_3$ \\
                    $a_3$ & $b_3$ & $c_3$ & $m_1$ & $n_1$ \\
                    $a_3$ & $b_3$ & $c_3$ & $m_2$ & $n_2$ \\
                    $a_3$ & $b_3$ & $c_3$ & $m_3$ & $n_3$ \\ \hline
                \end{tabular}
            }
        \end{center}
        \end{RelAlgSubsection}

    \end{CheatsheetEntryFrame}

\end{multicols}
\newpage
\begin{multicols}{2}

    \begin{CheatsheetEntryFrame}

        \CheatsheetEntryTitle{Theta Join (or Inner Join) and Equijoin}

        \textit{\myul{Theta join}} combines the tuples of two relations using a matching criterion.

        For $r(R)$ and $s(S)$, and some arbitrary matching criterion $c$:
        \begin{equation*}
            r \bowtie_c s = \setdef{(t_r : t_s)}{(t_r \in r) \wedge (t_s \in s) \wedge c(t_r : t_s)}
        \end{equation*}

        Theta join can also be defined as:
        \begin{equation*}
            r \bowtie_c s = \relselect_{c(r, s)}{\parens*{r \times s}}
        \end{equation*}
        Duplicate attributes are not removed.

        \vspace{\TextExtraSkip}%
        \textit{\myul{Equijoin}} is a special case of theta join that only tests for equality between attributes.

        \SubsectionFrameAddSeparation
        \begin{RelAlgSubsection}{Example}
        \begin{center}
            {\footnotesize%
                \begin{tabular}{|ccc|}
                    \multicolumn{3}{l}{\normalsize $r$}
                        \\ \hline
                    \multicolumn{1}{|c}{$A$}
                        & \multicolumn{1}{c}{$B$}
                        & \multicolumn{1}{c|}{$C$}
                        \\ \hline\hline
                    $a_1$ & $b_1$ & $0$ \\
                    $a_2$ & $b_2$ & $3$ \\
                    $a_3$ & $b_3$ & $2$ \\ \hline
                    \multicolumn{3}{c}{} \\ % Empty Gap
                    \multicolumn{3}{l}{\normalsize $s$}
                        \\ \cline{1-2}
                    \multicolumn{1}{|c}{$M$}
                        & \multicolumn{1}{c|}{$N$}
                        & \multicolumn{1}{c}{} % Empty cell
                    \\ \cline{1-2} \cline{1-2}
                        $m_1$ & \multicolumn{1}{c|}{$3$} & \multicolumn{1}{c}{} \\
                        $m_2$ & \multicolumn{1}{c|}{$4$} & \multicolumn{1}{c}{} \\
                        $m_3$ & \multicolumn{1}{c|}{$1$} & \multicolumn{1}{c}{} \\
                        $m_4$ & \multicolumn{1}{c|}{$2$} & \multicolumn{1}{c}{} \\ \cline{1-2}
                \end{tabular}
                \qquad
                \begin{tabular}{|ccccc|}
                    \multicolumn{5}{l}{\normalsize $r \bowtie_{N \le C} s$}
                        \\ \hline
                    \multicolumn{1}{|c}{$A$}
                        & \multicolumn{1}{c}{$B$}
                        & \multicolumn{1}{c}{$C$}
                        & \multicolumn{1}{c}{$M$}
                        & \multicolumn{1}{c|}{$N$}
                        \\ \hline\hline
                    $a_2$ & $b_2$ & $3$ & $m_1$ & $3$ \\
                    $a_2$ & $b_2$ & $3$ & $m_3$ & $1$ \\
                    $a_2$ & $b_2$ & $3$ & $m_4$ & $2$ \\
                    $a_3$ & $b_3$ & $2$ & $m_3$ & $1$ \\
                    $a_3$ & $b_3$ & $2$ & $m_4$ & $2$ \\ \hline
                \end{tabular}
            }
        \end{center}
        \end{RelAlgSubsection}

    \end{CheatsheetEntryFrame}
    
    \begin{CheatsheetEntryFrame}

        \CheatsheetEntryTitle{Natural Join}

        A special case of \textit{equijoin} that matches tuples on all their common attributes.

        For $r(R)$ and $s(S)$:
        \begin{gather*}
            r \bowtie s = \setdef{\parens*{t_r : t_s[S - R]}}{(t_r \in r) \wedge (t_s \in s) \wedge m(t_r, t_s)} \\
            \intertext{where $m$ is ``all common attributes must match":}
            %m = \bigwedge_{X \in \parens*{R \cap S}}{\parens*{r[X] = s[X]}} % Original "each-attribute" version
            m(t_r, t_s) = \parens*{\vphantom{\frac{x}{x}} t_r[R \cap S] = t_s[R \cap S]} % New in-line projection version
        \end{gather*}

        Natural join can also be defined as:
        \begin{gather*}
            r \bowtie s = \relproject_{R \cup S}{\parens*{\relselect_{m(r, s)}{\parens*{r \times s}}}} \\
            \intertext{where $m$ is:}
            %m = \bigwedge_{X \in \parens*{R \cap S}}{\parens*{r[X] = s[X]}} % Original "each-attribute" version
            m(r, s) = \parens*{\vphantom{\frac{x}{x}} r[R \cap S] = s[R \cap S]} % New in-line projection version
        \end{gather*}
        and $\relproject_{R \cup S}$ assumes removal of duplicate attributes.

        \vspace{\TextExtraSkip}%
        \textbf{\color{mycontrastred}
            \textit{Natural join} is dangerous in real applications.\\[0mm]
            Use \textit{equijoin} to explicitly match tuples instead.
        }

        \SubsectionFrameRemoveSeparation
        \begin{RelAlgSubsection}{Example}
        \begin{center}
            {\footnotesize%
                \begin{tabular}{|cccc|}
                    \multicolumn{4}{l}{\normalsize $r$}
                        \\ \hline
                    \multicolumn{1}{|c}{$A$}
                        & \multicolumn{1}{c}{$B$}
                        & \multicolumn{1}{c}{$C$}
                        & \multicolumn{1}{c|}{$D$}
                        \\ \hline\hline
                    $a_1$ & $b_1$ & $x$ & $x$ \\
                    $a_2$ & $b_2$ & $\memphRC{x}$ & $\memphRC{y}$ \\
                    $a_3$ & $b_3$ & $\memphRC{x}$ & $\memphRC{y}$ \\
                    $a_4$ & $b_4$ & $\memphBC{y}$ & $\memphBC{x}$ \\ \hline
                    \multicolumn{4}{c}{} \\ % Empty Gap
                    \multicolumn{4}{l}{\normalsize $s$}
                        \\ \cline{1-3}
                    \multicolumn{1}{|c}{$C$}
                        & \multicolumn{1}{c}{$D$}
                        & \multicolumn{1}{c|}{$E$}
                        & \multicolumn{1}{c}{} % Empty cell
                    \\ \cline{1-3} \cline{1-3}
                        $y$ & $y$ & \multicolumn{1}{c|}{$e_1$} & \multicolumn{1}{c}{} \\
                        $\memphRC{x}$ & $\memphRC{y}$ & \multicolumn{1}{c|}{$e_2$} & \multicolumn{1}{c}{} \\
                        $\memphBC{y}$ & $\memphBC{x}$ & \multicolumn{1}{c|}{$e_3$} & \multicolumn{1}{c}{} \\ \cline{1-3}
                \end{tabular}
                \quad
                \begin{tabular}{|ccccc|}
                    \multicolumn{5}{l}{\normalsize $r \bowtie s$}
                        \\ \hline
                    \multicolumn{1}{|c}{$A$}
                        & \multicolumn{1}{c}{$B$}
                        & \multicolumn{1}{c}{$C$}
                        & \multicolumn{1}{c}{$D$}
                        & \multicolumn{1}{c|}{$E$}
                        \\ \hline\hline
                    $a_2$ & $b_2$ & $\memphRC{x}$ & $\memphRC{y}$ & $e_2$ \\
                    $a_3$ & $b_3$ & $\memphRC{x}$ & $\memphRC{y}$ & $e_2$ \\
                    $a_4$ & $b_4$ & $\memphBC{y}$ & $\memphBC{x}$ & $e_3$ \\ \hline
                \end{tabular}
            }
        \end{center}
        \end{RelAlgSubsection}

    \end{CheatsheetEntryFrame}
    
\end{multicols}
\newpage
\begin{multicols}{2}

    \begin{CheatsheetEntryFrame}

        \CheatsheetEntryTitle{Division} \MarkExtendedRelAlg

        For $r(R)$ and $s(S)$, with $S \subseteq R$:
        \newcommand{\X}{\rule[-2ex]{0mm}{2ex}}
        \begin{gather*}
            r \div s = \setdef{t \in \relproject_{R-S}{(r)}}{\Phi(t)} \\
            \intertext{where:}
            %\Phi(t) = \forall t_s \in s \, \exists t_r \in r \, (t_r[S] = t_s \wedge t = t_r[R-S])
            \Phi(t) =
                \underbrace{\X \forall t_s \in s}_{(1)}
                \, \underbrace{\X \exists t_r \in r}_{(2)}
                \, (
                    \underbrace{\X t_r[S] = t_s}_{(3)}
                    \underbrace{\X \wedge t = t_r[R-S]}_{(4)}
                )
        \end{gather*}

        \textit{In plain English:} (1)~For all tuples in $s$, (2)~there is at least one relation in $r$ (3)~whose common attributes match, (4)~and whose non-common attributes are the same.

        \vspace{\TextExtraSkip}%
        Division can also be defined as:
        \begin{gather*}
            r \div s =
                r' -
                \underbrace{ \rule[-2ex]{0mm}{2ex}
                    \relproject_{(R-S)}{\parens*{\vphantom{\frac{x}{x}} [r' \times s] - r}}
                }_{\text{``disqualifier" term}}
                \\
            \intertext{where $r'$ is ``all possible result tuples":}
            r' = \relproject_{(R-S)}{(r)}
        \end{gather*}

        \SubsectionFrameRemoveSeparation
        \begin{RelAlgSubsection}{Example}
        \begin{center}
            {\footnotesize%
                \newcommand{\FmtA}[1]{\memphBC{#1}} % Alternating formatting
                \newcommand{\FmtB}[1]{\memphRC{#1}}
                \begin{tabular}{|cccc|}
                    \multicolumn{4}{l}{\normalsize $r$}
                        \\ \hline
                    \multicolumn{1}{|c}{$A$}
                        & \multicolumn{1}{c}{$B$}
                        & \multicolumn{1}{c}{$C$}
                        & \multicolumn{1}{c|}{$D$}
                        \\ \hline\hline
                    $      \alpha   $ & $      \beta  $ & $      x $ & $      x $ \\
                    $\FmtA{\alpha  }$ & $\FmtA{\beta }$ & $\FmtA{x}$ & $\FmtA{z}$ \\
                    $\FmtA{\alpha  }$ & $\FmtA{\beta }$ & $\FmtA{w}$ & $\FmtA{z}$ \\
                    $      \gamma   $ & $      \delta $ & $      x $ & $      x $ \\
                    $\FmtB{\gamma  }$ & $\FmtB{\delta}$ & $\FmtB{x}$ & $\FmtB{y}$ \\
                    $\FmtB{\gamma  }$ & $\FmtB{\delta}$ & $\FmtB{x}$ & $\FmtB{z}$ \\
                    $      \gamma   $ & $      \delta $ & $      w $ & $      x $ \\
                    $      \gamma   $ & $      \delta $ & $      w $ & $      y $ \\
                    $\FmtB{\gamma  }$ & $\FmtB{\delta}$ & $\FmtB{w}$ & $\FmtB{z}$ \\
                    $      \gamma   $ & $      \delta $ & $      w $ & $      w $ \\
                    $\FmtA{\epsilon}$ & $\FmtA{\zeta }$ & $\FmtA{x}$ & $\FmtA{y}$ \\
                    $\FmtA{\epsilon}$ & $\FmtA{\zeta }$ & $\FmtA{x}$ & $\FmtA{z}$ \\
                    $\FmtA{\epsilon}$ & $\FmtA{\zeta }$ & $\FmtA{w}$ & $\FmtA{z}$ \\
                    $\FmtB{\eta    }$ & $\FmtB{\theta}$ & $\FmtB{x}$ & $\FmtB{y}$ \\ \hline
                \end{tabular}
                \quad
                \begin{tabular}{|cc|}
                    \multicolumn{2}{l}{\normalsize $s$}
                        \\ \hline
                    \multicolumn{1}{|c}{$C$}
                        & \multicolumn{1}{c|}{$D$}
                    \\ \hline \hline
                    $x$ & $y$ \\
                    $x$ & $z$ \\
                    $w$ & $z$ \\ \hline
                \end{tabular}
                \qquad\quad
                \begin{tabular}{|cc|}
                    \multicolumn{2}{l}{\normalsize $r \div s$}
                        \\ \hline
                    \multicolumn{1}{|c}{$A$}
                        & \multicolumn{1}{c|}{$B$}
                        \\ \hline\hline
                    $\gamma  $ & $\delta$ \\
                    $\epsilon$ & $\zeta $ \\ \hline
                \end{tabular}
            }
        \end{center}
        \medskip
        {\footnotesize%
            In $r$, values $(\gamma, \delta)$ and $(\epsilon, \zeta)$ are the only values that occur with all values $(x, y)$, $(x, z)$, and $(w, z)$.
        }
        \end{RelAlgSubsection}

    \end{CheatsheetEntryFrame}

    \begin{CheatsheetEntryFrame}

        \newcommand{\RelAlgVennDiagram}[3]{%
            \begin{tikzpicture}[scale=0.4, transform shape]
                \begin{scope} % Left
                    \clip
                        (-1, -1) rectangle (2.2, 1)
                        (1.2, 0) circle (1)
                    ;
                    \fill[#1] (0, 0) circle (1);
                \end{scope}
                \begin{scope} % Middle
                    \clip (1.2, 0) circle (1);
                    \fill[#2] (0, 0) circle (1);
                \end{scope}
                \begin{scope} % Right
                    \clip
                        (-1, -1) rectangle (2.2, 1)
                        (0, 0) circle (1)
                    ;
                    \fill[#3] (1.2, 0) circle (1);
                \end{scope}
                \begin{scope} % Outline
                    \draw[line width=2.0pt]
                        (1.2, 0) circle (1)
                        (0, 0) circle (1)
                    ;
                \end{scope}
            \end{tikzpicture}%
        }

        \CheatsheetEntryTitle{Outer Join} \MarkExtendedRelAlg

        Similar to \textit{theta join}, except in the result relation, we also include tuples that don't have matches in the join.

        These unmatched tuples get padded with null values in the result relation.

        %For each definition below, we consider two relations $r(R)$ and $s(S)$, and an arbitrary matching criterion $c$.
        %% Leaving this out for space reasons... but I think I'd rather keep it?

        \vspace{\TextExtraSkip}%
        \textit{\myul{Full outer join}} includes all tuples of both operands:
        \begin{align*}
            r \relfullouterjoin_c s &=
                (r \bowtie_c s)
                \cup \parens*{\vphantom{\frac{x}{x}}
                    \parens*{r - \relproject_R{(r \bowtie_c s)}}
                    \times \braces*{(\relnullvalue, \dots)}
                } \\
            &
                \qquad
                \cup \parens*{\vphantom{\frac{x}{x}}
                    \parens*{r - \relproject_R{(r \bowtie_c s)}}
                    \times \braces*{(\relnullvalue, \dots)}
                } \\
            &= (r \relleftouterjoin_c s) \cup (r \relrightouterjoin_c s)
        \end{align*}

        \textit{\myul{Left outer join}} includes all tuples of the left:
        \begin{align*}
            r \relleftouterjoin_c s &=
                (r \bowtie_c s)
                \cup \parens*{\vphantom{\frac{x}{x}}
                    \parens*{r - \relproject_R{(r \bowtie_c s)}}
                    \times \braces*{(\relnullvalue, \dots)}
                } \\
            &=
                s \relrightouterjoin_c r
        \end{align*}

        \textit{\myul{Right outer join}} includes all tuples of the right:
        \begin{align*}
            r \relrightouterjoin_c s &=
                (r \bowtie_c s)
                \cup \parens*{\vphantom{\frac{x}{x}}
                    \parens*{s - \relproject_S{(r \bowtie_c s)}}
                    \times \braces*{(\relnullvalue, \dots)}
                } \\
            &=
                s \relleftouterjoin_c r
        \end{align*}

        To summarize the differences between the joins:

        \vspace{1.0ex}
        \FourColumnsMinipages{%
            \footnotesize%
            \textsc{Inner/Theta}\\[0.5ex]
            \RelAlgVennDiagram{white}{mycontrastred}{white}
        }{%
            \footnotesize%
            \textsc{Left Outer}\\[0.5ex]
            \RelAlgVennDiagram{mycontrastred}{mycontrastred}{white}
        }{%
            \footnotesize%
            \textsc{Right Outer}\\[0.5ex]
            \RelAlgVennDiagram{white}{mycontrastred}{mycontrastred}
        }{%
            \footnotesize%
            \textsc{Full Outer}\\[0.5ex]
            \RelAlgVennDiagram{mycontrastred}{mycontrastred}{mycontrastred}
        }

        \SubsectionFrameAddSeparation
        \begin{RelAlgSubsection}{Example}
        \begin{center}
            {\footnotesize%
                \begin{tabular}{|cccc|}
                    \multicolumn{4}{l}{\normalsize $r$}
                        \\ \hline
                    \multicolumn{1}{|c}{$A$}
                        & \multicolumn{1}{c}{$B$}
                        & \multicolumn{1}{c}{$C$}
                        & \multicolumn{1}{c|}{$D$}
                        \\ \hline\hline
                    $a_1$ & $b_1$ & $x$ & $x$ \\
                    $a_2$ & $b_2$ & $\memphRC{x}$ & $\memphRC{y}$ \\
                    $a_3$ & $b_3$ & $\memphBC{y}$ & $\memphBC{x}$ \\
                    $a_4$ & $b_4$ & $z$ & $w$ \\ \hline
                    \multicolumn{4}{c}{} \\ % Empty Gap
                    \multicolumn{4}{l}{\normalsize $s$}
                        \\ \cline{1-3}
                    \multicolumn{1}{|c}{$C$}
                        & \multicolumn{1}{c}{$D$}
                        & \multicolumn{1}{c|}{$E$}
                        & \multicolumn{1}{c}{} % Empty cell
                    \\ \cline{1-3} \cline{1-3}
                        $y$ & $y$ & \multicolumn{1}{c|}{$e_1$} & \multicolumn{1}{c}{} \\
                        $\memphRC{x}$ & $\memphRC{y}$ & \multicolumn{1}{c|}{$e_2$} & \multicolumn{1}{c}{} \\
                        $\memphBC{y}$ & $\memphBC{x}$ & \multicolumn{1}{c|}{$e_3$} & \multicolumn{1}{c}{} \\ \cline{1-3}
                \end{tabular}
                \quad
                \begin{tabular}{|ccccc|}
                    \multicolumn{5}{l}{\normalsize $r \relfullouterjoin s$}
                        \\ \hline
                    \multicolumn{1}{|c}{$A$}
                        & \multicolumn{1}{c}{$B$}
                        & \multicolumn{1}{c}{$C$}
                        & \multicolumn{1}{c}{$D$}
                        & \multicolumn{1}{c|}{$E$}
                        \\ \hline\hline
                    $a_1$           & $b_1$           & $x$           & $x$           & $\relnullvalue$ \\
                    $a_2$           & $b_2$           & $\memphRC{x}$ & $\memphRC{y}$ & $e_2$           \\
                    $a_3$           & $b_3$           & $\memphBC{y}$ & $\memphBC{x}$ & $e_3$           \\
                    $a_4$           & $b_4$           & $z$           & $w$           & $\relnullvalue$ \\
                    $\relnullvalue$ & $\relnullvalue$ & $y$           & $y$           & $e_1$           \\ \hline
                    \multicolumn{5}{c}{} \\ % Empty Gap
                    \multicolumn{5}{l}{\normalsize $r \relleftouterjoin s$}
                        \\ \hline
                    \multicolumn{1}{|c}{$A$}
                        & \multicolumn{1}{c}{$B$}
                        & \multicolumn{1}{c}{$C$}
                        & \multicolumn{1}{c}{$D$}
                        & \multicolumn{1}{c|}{$E$}
                        \\ \hline\hline
                    $a_1$           & $b_1$           & $x$           & $x$           & $\relnullvalue$ \\
                    $a_2$           & $b_2$           & $\memphRC{x}$ & $\memphRC{y}$ & $e_2$           \\
                    $a_3$           & $b_3$           & $\memphBC{y}$ & $\memphBC{x}$ & $e_3$           \\
                    $a_4$           & $b_4$           & $z$           & $w$           & $\relnullvalue$ \\ \hline
                \end{tabular}
            }
        \end{center}
        \end{RelAlgSubsection}

    \end{CheatsheetEntryFrame}
    
\end{multicols}
\newpage
\begin{multicols}{2}

    \begin{CheatsheetEntryFrame}

        \CheatsheetEntryTitle{Grouping Operator} \MarkExtendedRelAlg

        Performs calculations over groups of tuples within a relation. Produces a relation containing the results.

        For $r(R)$ and operator subscript $G$:
        \begin{equation*}
            \relgroup_G{(R)}
        \end{equation*}
        where $G$ is a list containing:
        \begin{itemize}
            \item one or more attributes from $R$ to be taken as \textit{grouping attributes}, and
            \item one or more \textit{aggregate functions}, written in the form $\theta{(A, \dots)}$, where $\theta$ is an aggregate function to be applied to attributes $A, \dots$.
        \end{itemize}

        %The result relation has a schema with attributes $G$.
        %% TODO: Figure out a way to represent this concisely...

        Formally, aggregate functions can be considered to take a multiset (i.e. a set with duplicates) of values.

        Aggregate functions defined in SQL-92:
        \begin{center}
            $\text{AVG}$, $\text{MAX}$, $\text{MIN}$, $\text{SUM}$, $\text{COUNT}$
        \end{center}
        Most major DBMS implementations offer many more aggregate functions and allow user-defined functions.

        \SubsectionFrameAddSeparation
        \begin{RelAlgSubsection}{Example}
        \begin{center}
            {\footnotesize%
                \begin{tabular}{|cccc|}
                    \multicolumn{4}{l}{\normalsize $r$}
                        \\ \hline
                    \multicolumn{1}{|c}{$A$}
                        & \multicolumn{1}{c}{$B$}
                        & \multicolumn{1}{c}{$C$}
                        & \multicolumn{1}{c|}{$D$}
                        \\ \hline\hline
                    $x$ & $\alpha$ & $5$ & $7$ \\
                    $x$ & $\alpha$ & $7$ & $3$ \\
                    $x$ & $\beta $ & $1$ & $9$ \\
                    $x$ & $\beta $ & $5$ & $6$ \\
                    $y$ & $\alpha$ & $3$ & $1$ \\
                    $y$ & $\alpha$ & $2$ & $1$ \\
                    $y$ & $\beta $ & $3$ & $5$ \\
                    $z$ & $\alpha$ & $9$ & $4$ \\ \hline
                \end{tabular}

                \begin{tabular}{|cccc|}
                    \multicolumn{4}{l}{\normalsize $\relgroup_{A, \text{COUNT}(A), \text{MAX}(C), \text{MAX}(D)}{(r)}$}
                        \\ \hline
                    \multicolumn{1}{|c}{$A$}
                        & \multicolumn{1}{c}{$\text{COUNT}(A)$}
                        & \multicolumn{1}{c}{$\text{MAX}(C)$}
                        & \multicolumn{1}{c|}{$\text{MAX}(D)$}
                        \\ \hline\hline
                    $x$ & $4$ & $7$ & $9$ \\
                    $y$ & $3$ & $3$ & $5$ \\
                    $z$ & $1$ & $9$ & $4$ \\ \hline
                \end{tabular}

                \begin{tabular}{|cccc|}
                    \multicolumn{4}{l}{\normalsize $\relgroup_{A, B, \text{COUNT}(A), \text{SUM}(C)}{(r)}$}
                        \\ \hline
                    \multicolumn{1}{|c}{$A$}
                        & \multicolumn{1}{c}{$B$}
                        & \multicolumn{1}{c}{$\text{COUNT}(A)$}
                        & \multicolumn{1}{c|}{$\text{SUM}(C)$}
                        \\ \hline\hline
                    $x$ & $\alpha$ & $2$ & $12$ \\
                    $x$ & $\beta $ & $2$ & $6$ \\
                    $y$ & $\alpha$ & $2$ & $5$ \\
                    $y$ & $\beta $ & $1$ & $3$ \\
                    $z$ & $\alpha$ & $1$ & $9$ \\ \hline
                \end{tabular}
            }
        \end{center}
        \medskip
        {\footnotesize%
            \textit{The $\text{COUNT}()$ function is interesting in that it doesn't really matter which column you use.}
        }
        \end{RelAlgSubsection}

    \end{CheatsheetEntryFrame}

    \begin{CheatsheetEntryFrame}

        \CheatsheetEntryTitle{Generalized Projection} \MarkExtendedRelAlg

        We can extend the projection operator $\pi$ to also contain expressions for computation.

        \SubsectionFrameAddSeparation
        \begin{RelAlgSubsection}{Example}
        \begin{center}
            {\footnotesize%
                \begin{tabular}{|cccc|}
                    \multicolumn{4}{l}{\normalsize $r$}
                        \\ \hline
                    \multicolumn{1}{|c}{$A$}
                        & \multicolumn{1}{c}{$B$}
                        & \multicolumn{1}{c}{$C$}
                        & \multicolumn{1}{c|}{$D$}
                        \\ \hline\hline
                    $a_1$ & $2$ & $6$ & $d_1$ \\
                    $a_2$ & $8$ & $9$ & $d_2$ \\
                    $a_3$ & $3$ & $2$ & $d_3$ \\ \hline
                \end{tabular}
                \qquad \qquad
                \begin{tabular}{|cc|}
                    \multicolumn{2}{l}{\normalsize $\relproject_{A, B + C}{(r)}$}
                        \\ \hline
                    \multicolumn{1}{|c}{$A$}
                        & \multicolumn{1}{c|}{$B + C$}
                        \\ \hline\hline
                    $a_1$ & $8$ \\
                    $a_2$ & $17$ \\
                    $a_3$ & $5$ \\ \hline
                \end{tabular}
            }
        \end{center}
        \end{RelAlgSubsection}

    \end{CheatsheetEntryFrame}

\end{multicols}
\newpage
\begin{multicols}{2}

    \begin{CheatsheetEntryFrame}

        \CheatsheetEntryTitle{Functional Dependency}

    \end{CheatsheetEntryFrame}
    
\end{multicols}

