\documentclass{article}

%\usepackage[margin=2.5cm, bottom=3.0cm]{geometry}
\usepackage[margin=1.5cm, bottom=2.0cm]{geometry}

\usepackage{../_latex_includes/sharedpkg}

\def\TITLE{Random Math Solutions}

\setcounter{tocdepth}{3}
\setcounter{secnumdepth}{3}

\begin{document}

\thispagestyle{plain}
\MakeCustomTitle
\bigskip

%%%%%%%%%%%%%%%%%%%%%%%%%%%%%%%%%%%%%%%%%%%%%%%%%%%%%%%%%%%%%%%%%%%%%%%%%%%%%%%%%%%%%%%%%%%%%%%%%%%%
%%%%%%%%%%%%%%%%%%%%%%%%%%%%%%%%%%%%%%%%%%%%%%%%%%%%%%%%%%%%%%%%%%%%%%%%%%%%%%%%%%%%%%%%%%%%%%%%%%%%

\begin{QuestionFrame}
	Considering:
	\begin{equation*}
		\int{\sech{ax} \,\diff{x}}
			= \frac{1}{a} \tan^{-1}{\parens*{\sinh{ax}}}
			= \frac{2}{a} \tan^{-1}{\parens*{\tanh{\frac{ax}{2}}}}
	\end{equation*}
	
	Prove that:
	\begin{equation*}
		\frac{1}{a} \tan^{-1}{\parens*{\sinh{ax}}}
			= \frac{2}{a} \tan^{-1}{\parens*{\tanh{\frac{ax}{2}}}}
	\end{equation*}
\end{QuestionFrame}

Let us use the following identities for this proof:
\begin{gather*}
	\tan^{-1}{x} = 2 \tan^{-1}{\parens*{\frac{x}{1 + \sqrt{1 + x^2}}}} \\
	\sinh{x} = \frac{e^x - e^{-x}}{2} = \frac{e^{2x} - 1}{2 e^x} \\
	\tanh{x} = \frac{e^x - e^{-x}}{e^x + e^{-x}} = \frac{e^{2x} - 1}{e^{2x} + 1}
\end{gather*}

We start by rewriting LHS:
\begin{equation*}
	\lhs
		= \frac{1}{a} \tan^{-1}{\parens*{\sinh{ax}}}
		= \frac{2}{a} \tan^{-1}{\parens*{\frac{\sinh{ax}}{1 + \sqrt{1 + \sinh^2{ax}}}}} 
\end{equation*}

Now, to finish the proof, we will now show the following to be true:
\begin{equation*}
	\tanh{\frac{ax}{2}} = \frac{\sinh{ax}}{1 + \sqrt{1 + \sinh^2{ax}}}
\end{equation*}

To simplify further working, we will change the variable:
\begin{equation*}
	\tanh{x} = \frac{\sinh{2x}}{1 + \sqrt{1 + \sinh^2{2x}}}
\end{equation*}

Starting with the RHS:
\begin{align*}
	\rhs
	&= \frac{\frac{e^{4x} - 1}{2 e^{2x}}}{1 + \sqrt{1 + \parens*{\frac{e^{4x} - 1}{2 e^{2x}}}^2}}
	= \frac{e^{4x} - 1}{2 e^{2x} \parens*{1 + \sqrt{1 + \frac{e^{8x} - 2 e^{4x} + 1}{4 e^{4x}}}}}
	= \frac{e^{4x} - 1}{2 e^{2x} \parens*{1 + \sqrt{\frac{e^{8x} - 2 e^{4x} + 1 + 4 e^{4x}}{4 e^{4x}}}}}
	= \frac{e^{4x} - 1}{2 e^{2x} \parens*{1 + \sqrt{\frac{e^{8x} + 2 e^{4x} + 1}{4 e^{4x}}}}}
	\\
	&= \frac{e^{4x} - 1}{2 e^{2x} \parens*{1 + \sqrt{\frac{\parens*{e^{4x} + 1}^2}{4 e^{4x}}}}}
	= \frac{e^{4x} - 1}{2 e^{2x} \parens*{1 + \frac{e^{4x} + 1}{2 e^{2x}}}}
	= \frac{e^{4x} - 1}{2 e^{2x} + e^{4x} + 1}
	= \frac{\parens*{e^{2x} - 1} \parens*{e^{2x} + 1}}{\parens*{e^{2x} + 1}^2}
	= \frac{e^{2x} - 1}{e^{2x} + 1}
	= \tanh{x}
	= \lhs
\end{align*}

as required.

%%%%%%%%%%%%%%%%%%%%%%%%%%%%%%%%%%%%%%%%%%%%%%%%%%%%%%%%%%%%%%%%%%%%%%%%%%%%%%%%%%%%%%%%%%%%%%%%%%%%
%%%%%%%%%%%%%%%%%%%%%%%%%%%%%%%%%%%%%%%%%%%%%%%%%%%%%%%%%%%%%%%%%%%%%%%%%%%%%%%%%%%%%%%%%%%%%%%%%%%%

\end{document}
